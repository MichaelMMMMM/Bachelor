\section{Untersuchung der Kinetik}
Im nächsten Schritt wird die Kinematik des Systems untersucht, um die Bewegungsgleichungen zu erhalten. Hierfür werden zunächst die Drehmomente analysiert, welche auf das Würfelgehäuse und die Schwungmasse wirken. Letztere wird einerseits von dem Motormoment $\bs{T}^{R/M}\idx{M}$ beschleunigt, andererseits wirkt ein verzögerndes Reibmoment $\bs{T}^{R/M}\idx{R}$, welches als proportional zu der Winkelgeschwindigkeit $\dot{\psi}$ modelliert wird.
\begin{equation}
\bs{T}^{R/M} = \bs{T}^{R/M}\idx{M} + \bs{T}^{R/M}\idx{R} = [T\idx{M} - C_{\psi}(u\idx{2} - u\idx{1})]\cdot \bs{k}\idx{3}
\end{equation}
Das Motor- und Reibmoment wirkt an dem Verbindungspunkt zwischen Schwungmasse und Würfelgehäuse, weshalb die beiden den Würfelkörper in umgekehrter Richtung  beeinflussen. Zusätzlich wird das Würfelgehäuse von dem Gravitationsmoment $\bs{T}^{K/O}_G$ beschleunigt. Die Summe der Komponenten ergibt das resultierende Drehmoment 
\begin{equation}
\bs{T}^{K/O} = T^{K/O}\idx{G} - \bs{T}^{R/M}\idx{M} - \bs{T}^{R/M}\idx{R} = [m\cdot g\cdot l_C\cdot s_{\varphi} - T\idx{M} + C_{\psi}(u\idx{2} - u\idx{1})] \cdot \bs{k}\idx{3}\,.
\end{equation}
Um die Bewegungsgleichungen zu ermitteln, werden die generalisierten aktiven Kräfte $F_i$ benötigt. Hierfür werden die Skalarprodukte der Drehmomente, welche auf die Körper wirken, und die partiellen Winkelgeschwindigkeiten, des entsprechenden Körpers, berechnet. Die Summe über alle Körper ergibt die generalisierte aktive Kraft $F_i$ \cite[S. 99 f.]{KaneBook}.
\begin{equation}
\begin{split}
F\idx{1} &= \inProd{\bs{T}^{K/O}}{\vel{A}{\omega}{K}\idx{1}} + \inProd{\bs{T}^{R/M}}{\vel{A}{\omega}{R}\idx{1}}
 = m\cdot g\cdot l\idx{C}\cdot s_{\varphi} - T\idx{M} + C_{\psi}(u\idx{2} - u\idx{1})
\\
F\idx{2} &= \inProd{\bs{T}^{K/O}}{\vel{A}{\omega}{K}\idx{2}} + \inProd{\bs{T}^{R/M}}{\vel{A}{\omega}{R}\idx{2}} = T\idx{M} - C_{\psi}(u\idx{2} - u\idx{1})
\end{split}
\end{equation}
Die partiellen Geschwindigkeiten können als Bewegungsrichtungen der Körper verstanden werden. Somit geben die generalisierten Kräfte $F_i$ den Beitrag der Drehmomente in die jeweilige Bewegungsrichtung wieder.

Als Gegenstück zu den generalisierten aktiven Kräften müssen nun die generalisierten Trägheitskräfte $F^*_i$ ermittelt werden. Das resultierende Trägheitsmoment der Schwungmasse ergibt sich aus dem Produkt des Skalars $I_R$, welches das Massenträgheitsmoment des Körpers relativ zu $M$ beschreibt, und der Winkelbeschleunigung $\vel{A}{\alpha}{R}$.\footnote{Die hier verwendete Berechnung ist nur für die eindimensionale Bewegung gültig. Für die Berechnung der Trägheitsmomente wird im allgemeinen ein Trägheitstensor $\bs{I}$ verwendet.}
\begin{equation}
\bs{T}^{R/M}_* = -I\idx{R}\cdot \vecBS{K}{0}{0}{\dot{u}\idx{2}}
\end{equation}
Das Trägheitsmoment des Würfelkörpers lässt sich aus dessen Trägheitsskalar $I_K$ und der Winkelbeschleunigung $\vel{A}{\alpha}{K}$ berechnen. Die Größe $I\idx{K}$ ist die Summe des Massenträgheitsmoments des Würfelgehäuses $I^{GH/O}$ und des Einflusses der Schwungmasse, welche dabei als Punkt mit der Masse $m\idx{R}$ und dem Abstand $l\idx{MO}$ betrachtet wird.
\begin{equation}
\bs{T}^{K/O}_* = -\underbrace{(I^{GH/O} + m\idx{R}\cdot l^2_{MO})}_{\equiv I\idx{K}}\cdot \vecBS{K}{0}{0}{\dot{u}\idx1} = -I\idx{K}\cdot \vecBS{K}{0}{0}{\dot{u}\idx1}
\end{equation}
Die generalisierten Trägheitskräfte $F^*_i$ werden wieder aus der folgenden Summe von Skalarprodukten berechnet \cite[S. 124 ff.]{KaneBook}.
\begin{equation}
\begin{split}
F^*\idx{1} &= \inProd{\bs{T}^{K/O}_*}{\vel{A}{\omega}{K}\idx{1}} + \inProd{\bs{T}^{R/M}_*}{\vel{A}{\omega}{R}\idx{1}} = -I\idx{K}\cdot \dot{u}\idx{1} 
\\
F^*\idx{2} &= \inProd{\bs{T}^{K/O}_*}{\vel{A}{\omega}{K}\idx{2}} + \inProd{\bs{T}^{R/M}_*}{\vel{A}{\omega}{R}\idx{2}} = -I\idx{R}\cdot \dot{u}\idx{2} 
\end{split}
\end{equation}
Die Bewegunsgleichungen resultieren nun aus Kanes Gleichung $F_i + F^*_i = 0$ \cite[S. 158 ff.]{KaneBook}.
\begin{equation}
\begin{split}
F\idx{1} + F^*\idx{1} = 0 &\hspace{15pt}\leftrightarrow\hspace{15pt} I\idx{K}\cdot \dot{u}\idx{1} = m\cdot g\cdot l\idx{C}\cdot s_{\varphi} - T\idx{M} + C_{\psi}(u\idx{2} - u\idx{1}) 
\\
F\idx{2} + F^*\idx{2} = 0 &\hspace{15pt}\leftrightarrow\hspace{15pt} I\idx{R}\cdot \dot{u}\idx{2} = T\idx{M} - C_{\psi}(u\idx{2} - u\idx{1})
\end{split}
\end{equation}
Rückblickend kann Kanes Methodik in die folgenden Schritte unterteilt werden. Zunächst wird die Kinematik des Systems analysiert, wobei die Körper mit Hilfe von Bezugssystemen modelliert werden. Die Definition von generalisierten Geschwindigkeiten ermöglicht die Bestimmung der partiellen Geschwindigkeiten, wodurch die Bewegung des Systems in Betrag und Richtung unterteilt wird. Anschließend werden die partiellen Geschwindigkeiten genutzt um die wirkenden Kräfte und Trägheitskräfte in die Bewegungsrichtungen zu projizieren. Hieraus resultieren die generalisierten aktiven Kräfte und generalisierten Trägheitskräfte, welche mittels Kanes Gleichung auf die gesuchten Bewegungsgleichungen führen.

Um die Bedeutung der generalisierten Geschwindigkeiten zu verdeutlichen, wird eine alternative Definition gewählt und die zugehörigen Bewegungsgleichungen ermittelt.
\begin{equation}
\tilde{u}\idx1 = \tilde{u}\idx{K} \equiv \dot{\varphi} \hspace{35pt} \tilde{u}\idx2 = \tilde{u}\idx{R} \equiv \dot{\psi}
\end{equation}
\begin{equation}
\begin{split}
\vel{A}{\tilde{\omega}}{K}\idx{1} = \bs{k}\idx{3} &\hspace{35pt} \vel{A}{\tilde{\omega}}{K}\idx{2} = 0
\\
\vel{A}{\tilde{\omega}}{R}\idx{1} = 0 &\hspace{35pt} \vel{A}{\tilde{\omega}}{R}\idx{2} = \bs{k}\idx{3}
\end{split}
\end{equation}
\begin{equation}
\begin{split}
\tilde{F}\idx1 = m\cdot g\cdot l\idx{C}\cdot s_{\varphi} \hspace{35pt}& \tilde{F}\idx2 = T\idx{M} - C_{\psi}(u\idx2-u\idx1) \hspace{25pt}
\\
\tilde{F}^*\idx{1} = -I\idx{R}\cdot (\dot{\tilde{u}}\idx1 + \dot{\tilde{u}}\idx2) - I\idx{K}\cdot \dot{\tilde{u}}\idx1 \hspace{35pt}& \tilde{F}^*\idx2 = -I\idx{R}\cdot (\dot{\tilde{u}}\idx1 + \dot{\tilde{u}}\idx2)\hspace{10pt}
\end{split}
\end{equation}
\begin{align}
\tilde{F}\idx1 + \tilde{F}^*\idx1 = 0 &\hspace{15pt}\leftrightarrow\hspace{15pt} I_k\cdot \dot{\tilde{u}}\idx1 + I\idx{R}\cdot (\dot{\tilde{u}}\idx1+\dot{\tilde{u}}\idx2) = m\cdot g\cdot l\idx{C}\cdot s_{\varphi} \label{tilde_bwg_1}
\\
\tilde{F}\idx{2} + \tilde{F}^*\idx{2} = 0 &\hspace{15pt}\leftrightarrow\hspace{15pt} I\idx{R}\cdot (\dot{\tilde{u}}\idx{1}+\dot{\tilde{u}}\idx{2}) = T\idx{M} - C_{\psi}(\dot{\tilde{u}}\idx2 - \dot{\tilde{u}}\idx1) \label{tilde_bwg_2}
\end{align}
Dieses Beispiel zeigt, dass die Wahl der generalisierten Geschwindigkeiten eine direkte Auswirkung auf die Form der resultierenden Bewegungsgleichungen hat. Zwar kann die Letztere mittels $u\idx1 = \tilde{u}\idx1 \ ,\ u\idx2 = \tilde{u}\idx1 + \tilde{u}\idx2$ und dem Einsetzen der Gleichung (\ref{tilde_bwg_2}) in die Gleichung (\ref{tilde_bwg_1}) in die ursprüngliche Form überführt werden, allerdings stellt die Modellierung des Würfels auf einer Ecke einen Anwendungsfall dar, bei dem der Berechnungsaufwand durch eine geschickte Wahl der generalisierten Geschwindigkeiten deutlich reduziert werden kann.