\section{3D-Modell}
Das nächste Ziel besteht darin ein Regelungskonzept zu entwickeln, welches das Balancieren des Würfels auf einer Ecke ermöglicht. Hierfür werden drei Motor verwendet, wodurch das gesamte System über sechs Freiheitsgrade verfügt. Die Vorgehensweise erfolgt analog zu dem Reglerentwurf der Würfelseite. Somit besteht der erste Schritt in dem Entwurf eines mechanischen Modells, welches wiederum zu einer Zustandsraumdarstellung führt, die als Grundlage für den Reglerentwurf verwendet wird.
Der erste Schritt in der Modellbildung besteht in der Definition der Bezugssystem, welche zur Beschreibung der Systembewegung dienen. Der Ausgangspunkt ist das raumfeste Bezugssystem $A$, welches durch die drei Einheitsvektoren $\bs{a}_1$, $\bs{a}_2$ und $\bs{a}_3$ definiert wird. Das Würfelgehäuse verfügt über drei rotatorische Freiheitsgrade, welche durch die Winkel $\varphi_1$, $\varphi_2$ und $\varphi_3$ beschrieben werden. Durch die Rotation des Würfels um den Winkel $\varphi_1$ in Richtung des Vektors $\bs{a}_1$ entsteht das Hilfsbezugssystem $B$, das durch die Einheitsvektoren $\bs{b}_1$, $\bs{b}_2$ und $\bs{b}_3$ aufgespannt wird.
\begin{equation}
\presuper{A}{\bs{v}} = \presuper{B}{\Big( \pMat{A}{B} \cdot \presuper{A}{\bs{v}}\Big)} = \presuper{B}{\bs{v}}
\end{equation}
Die Rotation um den Winkel $\varphi_2$ in Richtung des Vektors $\bs{b}_2$ führt zu dem zweiten Hilfsbezugssystem $C$ mit den drei Einheitsvektoren $\bs{c}_1$, $\bs{c}_2$ und $\bs{c}_3$.
\begin{equation}
\presuper{B}{\bs{v}} = \presuper{C}{\Big( \pMat{B}{C} \cdot \presuper{B}{\bs{v}}\Big)} = \presuper{C}{\bs{v}}
\end{equation}
Die letzte Rotation des Würfels in Richtung von $\bs{c}_3$ um den Winkel $\varphi_3$ führt zu dem körperfesten Bezugssystem $K$, welches durch die drei Vektoren $\bs{k}_1$, $\bs{k}_2$ und $\bs{k}_3$ definiert ist.
\begin{equation}
\presuper{C}{\bs{v}} = \presuper{K}{\Big( \pMat{C}{K} \cdot \presuper{C}{\bs{v}}\Big)} = \presuper{K}{\bs{v}}
\end{equation}
Hier sei angemerkt, dass es sich bei den Bezugssystem $B$ und $C$ um theoretische Konstrukte handelt, für die kein physisches Gegenstück existiert. Sie werden lediglich als Hilfsmittel zur Beschreibung des Systems verwendet.

Durch die Rotation der Schwungmassen besitzt das System drei weitere Freiheitsgrade, welche von den Winkeln $\psi_1$, $\psi_2$ und $\psi_3$ beschrieben werden. Somit entstehen drei weitere Bezugssysteme, welche jeweils an den Schwungmassen fixiert sind. Allerdings spielen diese keine weitere Rolle, da es sich bei den Winkeln $\psi_i$ um zyklische Koordinaten handelt. Das heißt, dass der Impuls des Systems nicht von der Ausrichtung der Schwungmassen abhängt. Lediglich die Winkelgeschwindigkeiten $\dot{\psi}_i$ beeinflussen auf Grund der Reibung das System.

Die Position und Ausrichtung des Systems wird von den sechs Winkeln $\varphi_i$ und $\psi_i$ vollständig beschrieben. Deshalb werden diese als generalisierte Koordinaten $q_i$ definiert.
\begin{equation}
q_i = \varphi_i \hspace{35pt} q_j = \psi_i \hspace{35pt} (i=1,2,3; j=4,5,6)
\end{equation}
Mit Hilfe der Bezugssysteme und generalisierten Koordinaten können nun die Winkelgeschwindigkeit des Würfels $\vel{A}{\omega}{K}$ und der Schwungmassen $\vel{A}{\omega}{R_i}$ bestimmt werden. Diese ergeben sich aus der Addition der relativen Rotationsgeschwindigkeiten der Bezugssysteme zueinander.
\begin{equation}
\begin{split}
\vel{A}{\omega}{K} &= \vel{A}{\omega}{B}+\vel{B}{\omega}{C}+\vel{C}{\omega}{K} = \vecBS{A}{\dot{\varphi}_1}{0}{0} + \vecBS{B}{0}{\dot{\varphi}_2}{0} + \vecBS{C}{0}{0}{\dot{\varphi}_3} \\
&= \vecBS{K}
{\dot{\varphi}_2\cdot s_{\varphi_3} + \dot{\varphi}_1 \cdot c_{\varphi_2}\cdot c_{\varphi_3}}
{\dot{\varphi}_2\cdot c_{\varphi_3} - \dot{\varphi}_1 \cdot c_{\varphi_2}\cdot s_{\varphi_3}}
{\dot{\varphi}_3 + \dot{\varphi}_1\cdot s_{\varphi_2}}
\end{split}
\end{equation}
Die Winkelgeschwindigkeiten der Schwungmassen $\vel{K}{\omega}{R_i}$ relativ zu dem Würfel entsprechen der ersten Ableitung der Winkel $\psi_i$. Mit Hilfe des Additionstheorems für Winkelgeschwindigkeiten kann daraus auch die absolute Winkelgeschwindigkeit der Schwungmassen $\vel{A}{\omega}{R_i}$ berechnet werden.
\begin{equation}
\vel{K}{\omega}{R_1} = \vecBS{K}{\dot{\psi}_1}{0}{0} \hspace{35pt}
\vel{K}{\omega}{R_2} = \vecBS{K}{0}{\dot{\psi}_2}{0} \hspace{35pt}
\vel{K}{\omega}{R_3} = \vecBS{K}{0}{0}{\dot{\psi}_3} 
\end{equation}
\begin{equation}
\vel{A}{\omega}{R_i} = \vel{A}{\omega}{K} + \vel{K}{\omega}{R_i} \hspace{35pt} (i=1,2,3)
\end{equation}
Im nächsten Schritt werden die absoluten Geschwindigkeiten der Teilsysteme in Komponenten zerlegt, welche sich aus den generalisierten Geschwindigkeiten $u_i$ und partiellen Geschwindigkeiten $\vel{A}{\omega}{j}_i$ zusammensetzen. Hierfür wird die folgende Definition der für die generalisierten Geschwindigkeiten verwendet.
\begin{equation}
\begin{split}
u_1 &= \dot{\varphi}_2\cdot s_{\varphi_3} + \dot{\varphi}_1\cdot c_{\varphi_2}\cdot c_{\varphi_3} \\
u_2 &= \dot{\varphi}_2\cdot c_{\varphi_3} - \dot{\varphi}_1\cdot c_{\varphi_2}\cdot s_{\varphi_3} \\
u_3 &= \dot{\varphi}_3 + \dot{\varphi}_1\cdot s_{\varphi_2} \\
u_4 &= \dot{\psi}_1 \hspace{35pt} u_5 = \dot{\psi}_2 \hspace{35pt} u_6 = \dot{\psi}_3
\end{split}
\end{equation}
Die Einführung der generalisierten Geschwindigkeiten führt einerseits zu einem stark vereinfachten Ausdruck der absoluten Winkelgeschwindigkeiten. Andererseits können dadurch auch die partiellen Geschwindigkeiten $\vel{A}{\omega}{j}_i$ in einfachen Termen ausgedrückt werden, wie die folgenden Gleichungen zeigen. Dadurch werden die kommenden Schritte der Modellbildung zunehmend erleichtert.
\begin{align}
\vel{A}{\omega}{K} &= u_1 \cdot \bs{k}_1 + u_2 \cdot \bs{k}_2 + u_3 \cdot \bs{k}_3 &\rArrow &\vel{A}{\omega}{K}_1 = \bs{k}_1, \vel{A}{\omega}{K}_2 = \bs{k}_2, \vel{A}{\omega}{K}_3 = \bs{k}_3 \\
& & &\vel{A}{\omega}{K}_4 = 0, \vel{A}{\omega}{K}_5 = 0, \vel{A}{\omega}{K}_6 = 0 
\\
\vel{A}{\omega}{R_1} &= u_1 \cdot \bs{k}_1 + u_2 \cdot \bs{k}_2 + u_3 \cdot \bs{k}_3 + u_4 \cdot \bs{k}_1 &\rArrow 
&\vel{A}{\omega}{K}_1 = \bs{k}_1, \vel{A}{\omega}{K}_2 = \bs{k}_2, \vel{A}{\omega}{K}_3 = \bs{k}_3 \\
& & &\vel{A}{\omega}{K}_4 = \bs{k}_1, \vel{A}{\omega}{K}_5 = 0, \vel{A}{\omega}{K}_6 = 0 
\\
\vel{A}{\omega}{R_2} &= u_1 \cdot \bs{k}_1 + u_2 \cdot \bs{k}_2 + u_3 \cdot \bs{k}_3 + u_5 \cdot \bs{k}_2 &\rArrow 
&\vel{A}{\omega}{K}_1 = \bs{k}_1, \vel{A}{\omega}{K}_2 = \bs{k}_2, \vel{A}{\omega}{K}_3 = \bs{k}_3 \\
& & &\vel{A}{\omega}{K}_4 = 0, \vel{A}{\omega}{K}_5 = \bs{k}_2, \vel{A}{\omega}{K}_6 = 0 
\\
\vel{A}{\omega}{R_3} &= u_1 \cdot \bs{k}_1 + u_2 \cdot \bs{k}_2 + u_3 \cdot \bs{k}_3 + u_6 \cdot \bs{k}_3 &\rArrow 
&\vel{A}{\omega}{K}_1 = \bs{k}_1, \vel{A}{\omega}{K}_2 = \bs{k}_2, \vel{A}{\omega}{K}_3 = \bs{k}_3 \\
& & &\vel{A}{\omega}{K}_4 = 0, \vel{A}{\omega}{K}_5 = 0, \vel{A}{\omega}{K}_6 = \bs{k}_3
\end{align}