\section{Systemparameter}\label{TM_3D_Systemparameter}
Zunächst werden die Parameter des mechanischen Systems vorgestellt. Das System setzt sich aus drei Schwungmassen und dem Würfelkörper zusammen. Unter dem Würfelkörper ist das Würfelgehäuse inklusive der montierten Motoren, Sensoren und Elektrik zu verstehen und wird mit $K$ bezeichnet. Für die Herleitung der Bewegungsgleichungen gilt die Annahme, dass der Würfelkörper nicht translativ bewegt wird, sondern lediglich um Punkt $O$ rotiert, bei dem es sich um die Ecke handelt, auf welcher der Würfel balanciert. Des Weiteren beschreiben alle Ortsvektoren den Vektor von $O$ zu dem jeweiligen Zielpunkt. Die drei Schwungmassen $R_i$ sind mit jeweils einem rotatorischem Freiheitsgrad auf den Motorwellen gelagert. Die Position der Lagerung wird mit $M_i$ bezeichnet und fällt auf Grund des symmetrischen Aufbaus der Schwungmassen mit deren Schwerpunkt zusammen.
Die Massen
\begin{equation}
m_R = 0{,}155\text{ kg}
\end{equation}
und Trägheitstensoren $\bs{I}^{Ri/Mi}$ der Schwungmassen werden mit Hilfe der CAD-Anwendung ermittelt. 
Letztere werden dabei aus der Perspektive des körperfesten Bezugssystem $K$ relativ zu den Punkten $M_i$ bestimmt. Für den Trägheitstensor der Schwungmasse $R_1$ ergibt sich der Wert
\begin{equation}
 \bs{I}^{R1/M1} = \begin{bmatrix}
3{,}358\cdot 10^{-4} & 2{,}641\cdot 10^{-11} & 0 
\\
2{,}651\cdot 10^{-11} & 1{,}961\cdot 10^{-4} & 4{,}527\cdot 10^{-9} 
\\
0 & 4{,}527\cdot 10^{-9} & 1{,}691\cdot 10^{-4}
\end{bmatrix} \text{ kg}\cdot \text{m}^2 \,.
\end{equation}
Hieran zeigt sich, dass die Vektorbasis des Bezugssystem $K$ nahezu den Haupträgheitsachsen der Schwungmasse entspricht, da die Devitationsmomente um die Größenordnung $10^{5}$ unterhalb der Haupträgheitsmomente liegen. Deshalb wird die folgende Untersuchung unter der Annahme durchgeführt, dass die Devitationsmomente vernachlässigt werden könne und somit gilt
\begin{equation}
\begin{split}
\bs{I}^{R1/M1} &\equiv \begin{bmatrix}
I^{R1}\idx{11} & 0 & 0 \\ 0 & I^{R1}\idx{22} & 0 \\ 0 & 0 & I^{R1}\idx{33}
\end{bmatrix} = 
\begin{bmatrix}
3{,}358\cdot 10^{-4} & 0 & 0 \\
0 & 1{,}961\cdot 10^{-4} & 0 \\
0 & 0 & 1{,}691\cdot 10^{-4}
\end{bmatrix} \text{ kg}\cdot \text{m}^2
\\
\bs{I}^{R2/M2} &\equiv \begin{bmatrix}
I^{R2}\idx{11} & 0 & 0 \\ 0 & I^{R2}\idx{22} & 0 \\ 0 & 0 & I^{R2}\idx{33}
\end{bmatrix} = 
\begin{bmatrix}
1{,}691\cdot 10^{-4} & 0 & 0 \\
0 & 3{,}358\cdot 10^{-4} & 0 \\
0 & 0 & 1{,}961\cdot 10^{-4}
\end{bmatrix} \text{ kg}\cdot \text{m}^2
\\
\bs{I}^{R3/M3} &\equiv \begin{bmatrix}
I^{R3}\idx{11} & 0 & 0 \\ 0 & I^{R3}\idx{22} & 0 \\ 0 & 0 & I^{R3}\idx{33}
\end{bmatrix} = 
\begin{bmatrix}
1{,}961\cdot 10^{-4} & 0 & 0 \\
0 & 1{,}691\cdot 10^{-4} & 0 \\
0 & 0 & 3{,}358\cdot 10^{-4}
\end{bmatrix} \text{ kg}\cdot \text{m}^2 \,.
\end{split}
\end{equation}
Für die Masse des Würfelkörpers $m\idx{K}$ und des Gesamtsystems $m$ ergibt sich
\begin{equation}
m\idx{K} = 1{,}07 \text{ kg} \hspace{35pt} m = m_K + 3\cdot m_R = 1{,}532 \text{ kg}\,.
\end{equation}
Bei der Berechnung des Trägheittensors $\bs{I}^{GH/0}$ des Würfelkörpers um den Punkt $O$ wird der Einfluss der Schwungmassen nicht beachtet. Dies erfolgt bei der Berechnung der Trägheitsmomente in den folgenden Abschnitten. Somit ergibt sich für den Trägheitstensor
\begin{equation}
\begin{split}
\bs{I}^{GH/O} &= \begin{bmatrix}
I^{GH}\idx{11} & I^{GH}\idx{12} & I^{GH}\idx{13} \\
I^{GH}\idx{21} & I^{GH}\idx{22} & I^{GH}\idx{23} \\
I^{GH}\idx{31} & I^{GH}\idx{32} & I^{GH}\idx{33}
\end{bmatrix} \\
&=
\begin{bmatrix}
1{,}520\cdot 10^{-2} & -5{,}201\cdot 10^{-3} & 5{,}375\cdot 10^{-3} \\
-5{,}201\cdot 10^{-3} & 1{,}52\cdot 10^{-2} & 5{,}225\cdot 10^{-3} \\
5{,}375\cdot 10^{-3} & 5{,}225\cdot 10^{-3} & 1{,}542\cdot 10^{-2}
\end{bmatrix}\text{ kg}\cdot \text{m}^2 \,.
\end{split}
\end{equation}
Der Ortsvektor $\bs{c}$ des Schwerpunkts des Gesamtsystems wird ebenfalls numerisch ermittelt. Da sich die Komponenten des Ortsvektors $\bs{c}$ lediglich um $10^{-1}\text{ mm}$ unterscheiden, werden diese als identisch betrachtet.
\begin{equation}
\bs{c} = \vecBS{K}{-6{,}61}{-6{,}60}{-6{,}57}\text{ cm} \approx \vecBS{K}{l\idx{C}}{l\idx{C}}{l\idx{C}} \hspace{15pt} \vert \hspace{15pt} l\idx{C} = 6{,}6\text{ cm}
\end{equation}
Des Weiteren entsteht durch die Bewegung der Schwungmassen ein Reibmoment, welches als proportional zu den Winkelgeschwindigkeiten der Schwungmassen modelliert wird. Für Proportionalitätsfaktor $C_{\psi}$ wurde experimentell der  Wert 
\begin{equation}
C_{\psi} = 3{,}1176\cdot 10^{-5}\text{ kg}\cdot \text{m}^2 \cdot \text{s}^{-1}
\end{equation}
ermittelt.