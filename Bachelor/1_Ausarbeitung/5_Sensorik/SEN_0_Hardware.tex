\section{Erfassung der Winkelgeschwindigkeiten $u_{Ki}$ und $u_{Ri}$}
Für die Berechnung des Reglers ist der Zustandsvektor $\bs{x}$ erforderlich. Dieser setzt sich aus den Vektoren
\begin{equation}
\bs{\varphi} = \begin{bmatrix}
\varphi\idx1 \\ \varphi\idx2 \\ \varphi\idx3
\end{bmatrix}, \hspace{15pt}
\bs{u}\idx{K} = \begin{bmatrix}
u\idx{K1} \\ u\idx{K2} \\ u\idx{K3}
\end{bmatrix}
, \hspace{15pt}
\bs{u}\idx{R} = \begin{bmatrix}
u\idx{R1} \\ u\idx{R2} \\ u\idx{R3}
\end{bmatrix}
\end{equation}
zusammen.\footnote{Für die Erfassung der Zustandsgrößen werden die Winkel $\varphi_i$ betrachtet. Der Regler verwendet allerdings die Abweichung $\overline{\varphi}_i$ der Winkel vom Arbeitspunkt. Deshalb wird der Arbeitspunkt von den Winkelwerten abgezogen bevor der Regler berechnet wird.} Um die Zustandsgrößen zu messen bzw. zu schätzen, sind in dem Würfelgehäuse sechs MPU9050-Module montiert \cite{MPU9250}, welche sowohl einen Beschleunigungs- als auch einen Drehratensensor besitzen. Die Drehratensensoren erfassen deren Winkelgeschwindigkeiten in die Richtung dreier Messachsen, welche paarweise orthogonal zueinander stehen. Die drei Messwerte werden zu einem Vektor $\bs{\omega}_i$ zusammengefasst. Um die Sensoren in das mechanische Modell zu integrieren, werden deren Messachsen als Vektorbasis des Bezugssystem $S_i, i \in \left\{1, 2, 3, 4, 5, 6\right\}$ definiert. Somit sind die Anzeigewerte der Drehratensensoren die Komponenten der absoluten Winkelgeschwindigkeiten aus Perspektive von $S_i$
\begin{equation}
\bs{\omega}_i = \begin{bmatrix}
\omega_{i\text{x}} \\ \omega_{i\text{y}} \\ \omega_{i\text{z}}
\end{bmatrix} = \vecBS{S_i}{\omega_{i\text{x}}}{\omega_{i\text{y}}}{\omega_{i\text{z}}}\,.
\end{equation}
Da die Sensoren auf dem Würfelkörper fixiert sind, stimmen deren absolute Winkelgeschwindigkeiten mit der des Würfelkörpers überein.
\begin{equation}
\vel{A}{\omega}{S_i} = \vel{A}{\omega}{K} = \vecBS{K}{u\idx{K1}}{u\idx{K2}}{u\idx{K3}}
\end{equation}
Somit müssen lediglich die Abbildungmatrizen $\pMat{S_i}{K}$ bestimmt werden, um aus den Messwerten die Zustandsgrößen $\bs{u}_K$ zu berechnen. Da die Sensoren so montiert sind, dass deren Messachsen parallel zu den Würfelkanten liegen, können die Projektionsmatrizen an Hand des Aufbaus bestimmt werden.\footnote{Hier wird die Annahme getroffen, dass die Messachsen parallel zu den Würfelkanten stehen. In der Realität enstehen durch Fertigungstoleranzen Abweichungen, welche hier nicht berücksichtigt werden.}
\begin{equation}
\pMat{S_{1,2}}{K} = \begin{bmatrix}
0 & 1 & 0 \\ 1 & 0 & 0\\ 0 & 0 & 1
\end{bmatrix}, \hspace{15pt}
\pMat{S_{3,4}}{K} = \begin{bmatrix}
0 & 0 & 1 \\ 0 & 1 & 0\\ 1 & 0 & 0
\end{bmatrix}, \hspace{15pt}
\pMat{S_{5,6}}{K} = \begin{bmatrix}
1 & 0 & 0 \\ 0 & 0 & 1\\ 0 & 1 & 0
\end{bmatrix}
\end{equation}
Der letztendliche Messwert $\bs{u}_K$ wird durch den Mittelwert der sechs Drehratensensoren gebildet.
\pagebreak

Die Zustände $u_{Ri}$ ergeben sich aus der Summe
\begin{equation}
u_{Ri} = u_{\text{K}i} + \dot{\psi}_i\,.
\end{equation}
Da die Größen $u_{\text{K}i}$ mittels der Drehratensensoren bestimmt werden, müssen im nächsten Schritt die Ableitungen $\dot{\psi}_i$ ermittelt werden. Diese stellen die relative Winkelgeschwindigkeiten der Schwungmassen in Drehrichtung der Motoren dar. Hierfür sind in den Motoren Hall-Sensoren integriert, welche von den Motortreibern ausgewertet werden. Die Motortreiber geben wiederum ein Spannungssignal aus, das proportional zu $\dot{\psi}_i$ ist. Diese Signale werden mit den Werten $u_{\text{K}i}$ addiert um daraus $u_{\text{R}i}$ zu erhalten.