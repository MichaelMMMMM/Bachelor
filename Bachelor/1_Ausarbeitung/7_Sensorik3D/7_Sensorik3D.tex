\section{Messung der Zustandsgrößen}
Die Berechnung der Stellgröße benötigt die aktuellen Werte des Zustandsvektors. Somit müssen Methoden entwickelt werden um diese Größen zu messen. Hierfür werden, analog zu dem Modell der Würfelseite, Beschleunigungs- und Drehratensensoren verwendet, welche die Beschleunigung und Winkelgeschwindigkeit des Würfelkörpers erfassen. Die Geschwindigkeiten der Schwungmassen wird wiederum durch die Auswertung eines analogen Signals ermittelt. 

Auf dem Würfelkörper sind insgesamt sechs 9250-Module montiert, welche  dreiachsige Beschleunigungs- und Drehratensensoren besitzen. Es werden jeweils zwei der Module parallel zu einem Einheitsvektor $\bs{k}_i$ angeordnet. Auf Grund dieser Ausrichtung entsprechen die Sensorachsen der Darstellung im Bezugssystem $K$. Folglich können die Anzeigewerte der Drehratensensoren leicht ermittelt werden.
\begin{equation}
\bs{\omega}_{ij} = \vecBS{K}{u_1}{u_2}{u_3} \hspace{35pt} (i \in {1, 2, 3}; j \in {1,2})
\end{equation}
Die Anzeigewerte $\bs{a}_{ij}$ der Beschleunigungssensoren setzten sich aus der resultierenden Beschleunigung $\vel{A}{a}{S_{ij}}$ und dem Erdbeschleunigungsvektor $\presuper{K}{\bs{g}}$aus Perspektive des Würfelkörpers zusammen.
\begin{equation}
\bs{a}_{ij} = \vel{A}{a}{S_{ij}} + \presuper{K}{\bs{g}}
\end{equation}
An dieser Stelle ist lediglich der Gravitationsanteil von Interesse, da dieser als Zustandsgröße in der Regelung eine Rolle spielt. Deshalb muss, ähnlich zu der Würfelseite, eine Berechnungvorschrift ermittelt werden um den Einfluss der resultierenden Beschleunigung zu eliminieren. Hierfür sei angenommen, dass die Ortsvektoren $\bs{s}_{ij}$ und $\bs{s}_{kn}$ zweier Sensoren und eine Diagonalmatirx $\bs{A}$ mit den folgenden Elementen gegeben sei.
\begin{equation}
\bs{s}_{ij} = \vecBS{K}{x_{ij}}{y_{ij}}{z_{ij}} \hspace{15pt}
\bs{s}_{kn} = \vecBS{K}{x_{kn}}{y_{kn}}{z_{kn}} \hspace{15pt}
\bs{A} = \begin{pmatrix} A_1 & 0 & 0 \\ 0 & A_2 & 0 \\ 0 & 0 & A_3 \end{pmatrix}
\end{equation}
Dann gilt für die Messgröße $\bs{y}=\bs{a}_{ij} - \bs{A}\cdot \bs{a}_{kn}$ der folgende Zusammenhang.
\begin{equation}
\begin{split}
\bs{y} &= \vel{A}{a}{S_{ij}} + \presuper{K}{\bs{g}} - \bs{A}\cdot \vel{A}{a}{S_{kn}} - \bs{A}\cdot \presuper{K}{\bs{g}} = [\vel{A}{a}{S_{ij}} - \bs{A}\cdot \vel{A}{a}{S_{kn}}] + [\bs{I} - \bs{A}]\cdot \presuper{K}{\bs{g}}
\end{split}
\end{equation}
Zunächst wird nur der erste Summand dieser Summe betrachtet.
\begin{equation}
\begin{split}
\vel{A}{a}{S_{ij}} - \bs{A}\cdot \vel{A}{a}{S_{kn}} &= \frac{\presuper{A}{d}}{dt}(\vel{A}{\omega}{K} \times \bs{s}_{ij}) - \bs{A} \cdot \frac{\presuper{A}{d}}{dt}(\vel{A}{\omega}{K} \times \bs{s}_{kn}) \\
&= [\vel{A}{\alpha}{K}\times \bs{s}_{ij} + \vel{A}{\omega}{K}\times \vel{A}{v}{S_{ij}}] - \bs{A}\cdot [\vel{A}{\alpha}{K}\times \bs{s}_{kn} + \vel{A}{\omega}{K}\times \vel{A}{v}{S_{kn}}]
\end{split}
\end{equation}
