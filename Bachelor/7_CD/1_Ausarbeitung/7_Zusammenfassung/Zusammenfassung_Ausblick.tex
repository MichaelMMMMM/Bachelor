\chapter{Zusammenfassung und Ausblick}
Die Aufgabe der vorliegenden Arbeit bestand in der Entwicklung und Implementierung eines Regelungkonzeptes, um einen Würfel auf einer seiner Ecken zu balancieren. Diese Aufgabenstellung wurde vollständig erfüllt, wobei das Endergebnis aus einer Anwendung besteht, welche die Sensorik auswertet, die entwickelten Filter- und Regleralgorithmen berechnet und die Aktorik entsprechend ansteuert, um den Würfel zu stabilisieren. Zusätzlich wurde eine Desktop-Applikation entworfen, welche mit der Zielplattform verbunden ist und sowohl die Konfiguration des Reglers als auch die Visualisierung relevanter Versuchsdaten ermöglicht. Die für dieses Ergebnis erforderlichen Aufgaben lassen sich in zwei Gruppen unterteilen. Auf der einen Seite steht die modellbasierte Entwicklung der nötigen Filter- und Regelungsalgorithmen. Auf der anderen Seite steht die Implementierung des Konzeptes auf einem BeagleBone Black, um deren Fähigkeit nachzuweisen.
Der erste Schritt besteht darin, die Bewegungsgleichungen herzuleiten, wofür Kanes Methodik verwendet wurde. Die linearisierten Bewegungsgleichungen wurden anschließend in eine Zustandsraumdarstellung überführt, um einen LQ-Regler zu entwerfen, wobei sowohl die nicht steuerbaren als auch nicht beobachtbaren Zustände des Systems beachtet werden müssen. Außerdem muss die Stellgrößenbeschränkung bei dem Reglerentwurf berücksichtigt werden, um Direktionalitätsprobleme zu vermeiden.
Die nächste Teilaufgabe bestand in der Erfassung der Zustandsgrößen mithilfe der vorhandenen Sensorik. Hierbei stellt die Bestimmung der Winkel $\varphi_i$, welche die Ausrichtung des Würfels beschreiben, ein besonderes Problem dar, da diese lediglich an Hand des Erdbeschleunigungvektors geschätzt werden können. Als Lösung wurde ein Algorithmus vorgestellt, der es ermöglicht die resultierenden Beschleunigungssignale aus den Messwerten der Beschleunigungssensoren zu eliminieren. Des Weiteren wurde ein Komplementärfilter implementiert, um die Güte der Winkelsignale zu erhöhen.

Der zweite Aufgabenteil besteht in der Implementierung des Regelungskonzeptes, wobeidas Ziel verfolgt wurde eine Infrastruktur zu schaffen, welche es ermöglicht, während des Entwicklungsprozesses Versuche effizient zu implementieren und durchzuführen. Um den Implementierungsaufwand von Kontroll- und Signalflüssen zu minimieren, wurde ein Ansatz der Template-Metaprogrammierung verfolgt. Im Anschluss wurde eine Komponentenarchitektur entworfen, um die Hauptaufgaben der Anwendung voneinander zu trennen. Hieraus ergibt sich der Vorteil, dass die Teile der Anwendung, welche für die Regelungstechnik relevant sind, priorisiert werden können und somit ein nahezu deterministisches Zeitverhalten resultiert. Zudem entsteht durch die Komponentenarchitektur eine übersichtliche Programmstruktur, die für beliebige mechatronische Anwendungen wiederverwendet werden kann. Hierbei muss lediglich der Kontroll- und Signalfluss der Versuche angepasst werden, was sich mit Hilfe der hier entwickelten Template-Methoden effizient realisieren lässt. Im letzten Schritt wurde eine Desktopapplikation entworfen, um mit der Zielplattform zu interagieren. Die Anwendung besteht aus einer graphischen Benutzeroberfläche, die einerseits relevante Versuchsdaten visualisiert und andererseits Bedienelemente bietet, um den Versuchsablauf zu konfigurieren. Für die Implementierung der Anwendung wurde die Bibliothek Qt verwendet.

An dieser Stelle sind weitere Verbesserungen möglich. Ein Ansatz besteht darin den TCP/IP-Server durch einen Webserver zu ersetzen. In dieser Konfiguration wird die Desktopanwendung als Website implementiert und mittels eines Browsers ausgeführt. Die Vorteile dieser Vorgehensweise bestehen darin, dass die Implementierung der Anwendung weniger Zeit in Anspruch nimmt. Außerdem handelt es sich um eine plattformunabhängige Lösung, so kann eine webbasierte Anwendung beispielsweise auch auf einem Tablet oder Smartphone ausgeführt werden. Dieser  Ansatz wurde im Rahmen dieser Arbeit ebenfalls verfolgt, wobei sich allerdings das Problem ergab, dass die verwendete JavaScript-Bibliothek \textit{Flot} die erforderlichen Datenmengen nicht in einer ausreichenden Geschwindigkeit darstellen kann. Insofern muss untersucht werden, welche Alternativen für die Visualisierung bestehen.

Auf der Seite der Modellbildung und Regelungstechnik sind ebenfalls Verbesserungen möglich. Das hier präsentierte Regelungskonzept ermöglicht das Balancieren des Würfels auf einer Ecke. Allerdings sind weitere Maßnahmen nötigt, um die verbleibenden Schwingungen zu reduzieren. Ein möglicher Ansatz besteht darin, eine Systemidentifikation durchzuführen. Einerseits kann dadurch das Modell verifiziert werden, andererseits können die Parameter der Strecke ermittelt und somit die Reglergüte verbessert werden. Hierbei ist zunächst zu entscheiden, ob ein kontinuierliches oder diskretes Modell für die Identifikation verwendet wird. Letzteres bringt den Nachteil mit sich, dass lediglich die Parameter der diskreten Übertragungsfunktion bestimmt werden, welche keinen physikalischen Größen entsprechen. Im Gegensatz dazu ermöglicht ein kontinuierliches Modell die Bestimmung der physikalischen Parameter \cite[S. 189 ff.]{UnbehauenSysId}. Jedoch ist die Identifikation kontinuierlicher Systeme mit einem Mehraufwand verbunden.
Als Versuch kann das hier entwickelte Regelungskonzept verwendet werden, um eine Identifikation im geschlossenen Regelkreis durchzuführen \cite[S. 126 ff.]{UnbehauenSysId}. Dadurch ist es möglich sowohl mit einen kontinuierlichen als auch diskreten Modell alle relevanten Parameter zu bestimmen. Allerdings stellen sowohl die Identifikation von Mehrgrößensystemen als auch die Identifikation im geschlossenen Regelkreis Probleme dar, welche nicht vollständig gelöst sind \cite[S. 187 f.]{UnbehauenSysId}. Ebenso ist zu beachten, dass die Systemidentifikation keine deutliche Verbesserung der Reglergüte garantiert. Mithilfe der Parameteridentifikation werden lediglich Probleme behoben, welche von Modellungenauigkeiten verursacht werden. Beispielsweise sind die Empfindlichkeit des geschlossenen Regelkreises gegenüber Störgrößen und Messfehler ebenso plausible Ursachen für die verbleibenden Schwingungen des Würfels. 
Bei dem Entwurf des Reglers wurden die Gewichtungsmatrizen so variiert, dass die Reglermatrix möglichst kleine Elemente enthält. Dadurch kann die Auswirkung der Stellgrößenbeschränkung reduziert werden. Wendet man diese Vorgehensweise bei dem Entwurf des Reglers für das Balancieren auf einer Kante an, so stellt sich bei diesem ebenfalls eine verbleibende Oszillation ein. Hieraus folgt die Vermutung, dass die Robustheit des Reglers durch die Adaption der Gewichtungsmatrix beeinflusst wird.
Aus diesem Grund besteht ein weiterer Ansatz zur Verbesserung des geschlossenen Regelkreises darin, die Reglerstruktur so anzupassen, dass diese möglichst unempfindlich gegenüber Störgrößen und Modellfehlern ist. Ein Beispiel hierfür stellt die $\mathcal{H}_\infty$-Regelung dar \cite[S. 224 ff.]{Ludyk}. Beispielsweise wird die $\mathcal{H}_\infty$-Regelung in \cite{Toda} genutzt, um mechanische Systeme, welche von starken Parameterschwankungen und Vibrationen betroffen sind, zu kontrollieren. 

Einen weiteren Ansatz stellt die modellprädikative Regelung dar \cite{MPC}, bei der zu jedem Abtastzeitpunkt ein, von dem aktuellen Zustand abhängendes, Optimierungsproblem gelöst wird, um die Reglermatrix neu zu berechnen. Bei diesem Vorgehen ist es möglich die Stellgrößenbeschränkung explizit zu beachten, wodurch die Direktionalitätsproblematik behoben werden sollte. Allerdings benötigt die Lösung des Optimierungsproblems eine beachtliche Rechenleistung, weshalb zu prüfen ist, ob das Konzept für die hier verwendete Zielplattform und Abtastfrequenz geeignet ist.

Neben dem Balancieren soll der Würfel auch in der Lage sein sich selbständig aufzurichten, wofür die Schwungmassen als Energiespeicher verwendet werden. Durch das Abbremsen der Schwungmassen wird deren Drehimpuls auf den Würfel übertragen und dieser springt auf. Um diesen Vorgang zu steuern, kann ein zweiter Regler entwickelt werden, welcher die Trajektorie des Würfels von seiner Ruhelage in die aufrechte Position kontrolliert. Sobald der Arbeitsbereich des hier vorgestellten Reglers erreicht ist, übernimmt dieser. Um ein solches Konzept zu ermöglichen, muss allerdings das Komplementärfilter überarbeitet werden, da dieses lediglich im Bereich des Arbeitspunktes gültig ist. Ein möglicher Ansatz besteht darin,  bei der Systemidentifikation eine Modellstruktur zu wählen, welche ein System- und Messrauschen beinhaltet (\cite{UnbehauenSysId}, S.59 ff.). Anhand dieses Modells kann ein Kalman-Filter entwickelt werden, um die Signalgüte zu erhöhen (\cite{KalmanFilter}). Ein Vorteil dieser Methode besteht darin, dass das Kalman-Filter auf das nicht lineare Systeme erweitert werden kann (\cite{AdamyNL}, S.503 ff.) und somit nicht auf einen Arbeitspunkt beschränkt ist.

Die Bearbeitung der weiteren Aufgaben wird durch diese Arbeit in der Hinsicht erleichtert, dass die Infrastruktur wiederverwendet werden kann. Beispielsweise können verschiedene Regler- und Filterkonzepte mithilfe der Komponentenarchitektur leicht integriert werden, weshalb der Fokus der folgenden Arbeit allein auf dem systemtheoretischen Teilgebiet liegen kann. Des Weiteren stellt diese Arbeit ein funktionierendes Gesamtkonzept vor. Somit können in weiteren Arbeiten einzelne Komponenten, wie zum Beispiel die Regler- oder Filtermethodik, bearbeitet und an dem bestehenden Gesamtsystem miteinander verglichen werden. Dadurch können die einzelnen Aufgaben parallel und auf ein Teilgebiet fokussiert bearbeitet werden.