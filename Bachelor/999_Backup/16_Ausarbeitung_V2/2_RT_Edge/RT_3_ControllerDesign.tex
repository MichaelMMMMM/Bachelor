\section{Entwurf von Zustandsregelungen}
Nachdem die Grundlagen der Zustandsraumdarstellung in den vorherigen Abschnitten diskutiert wurden, dient dieser Teil dem Entwurf des letztendlichen Reglers. Hierfür wird zunächst am Beispiel des Entwurfs durch Eigenwertvorgabe der Grundgedanke und die Vorteile der Zustandsregelung aufgezeigt. Anschließend wird ein Regler nach dem Prinzip der optimalen Regelung entworfen um den Würfel auf einer Kante zu stabilisieren. Diese Ergebnisse werden an der reellen Regelstrecke überprüft.

\subsection{Reglerentwurf durch Eigenwertvorgabe}
Der Grundgedanke der Zustandsregelung besteht darin den vollständigen Zustandsvektor $\bs{x}$ zurückzuführen und durch die Multiplikation mit einer Reglermatrix $\bs{K}$ den Stellgrößenvektor $\bs{u}$ zu ermitteln. In dieser Arbeit werden lediglich Regelkreise ohne Führungsvektor ($\bs{\omega}=0)$) verwendet, weshalb auf dessen Einfluss und den Entwurf des nötigen Vorfilters $\bs{V}$ nicht weiter eingegangen wird.

Aus der Abbildung gehen die Systemgleichungen des geschlossenen Regelkreises \dSS{A_g}{0}{C_g} hervor.
\begin{equation}
\bs{u} = -\bs{B_d}\bs{K}
\end{equation}
\begin{equation}
\textfrak{D}_g
: \left\{ \begin{array}{ll}
\bs{{x}}(k+1) = \bs{A}\cdot \bs{x}(k) - \bs{B}\bs{K}\cdot \bs{x}(k) = \underbrace{(\bs{A}-\bs{B}\bs{K})}_{\equiv \bs{A_g}}\cdot \bs{x}(k) \\
\bs{y}(k) = \underbrace{\bs{C}}_{\equiv \bs{C_g}}\cdot \bs{x}(k)
\end{array}
\right.
\end{equation}
Die Reglermatrix $\bs{K}$ ist nun so zu entwerfen, dass die gewünschten Eigenschaften des geschlossenen Regelkreises erreicht werden. Hierfür wird zunächst das SISO-System \dSS{A}{b}{c^T} betrachtet, wessen Systemmatrix $\bs{A}$ und Eingangsvektor $\bs{b}$ die folgende Form besitzt.
\begin{equation}
\bs{A} = \begin{bmatrix}
0 & 1 & 0 & \hdots & 0 \\
0 & 0 & 1 & \hdots & 0 \\
\vdots & \vdots & \vdots & \ddots & \vdots \\
-a_0 & -a_1 & -a_2 & \hdots & -a_{n-1}
\end{bmatrix}
\hspace{35pt}
\bs{b} = \begin{bmatrix}
0 \\ 0 \\ \vdots \\ 1
\end{bmatrix}
\end{equation}
Diese Darstellung heißt Regelungsnormalform und kann mittels einer Zustandstransformation erreicht werden. Diese Transformation wird erst in einem späteren Abschnitt erläutert, da das Verfahren lediglich als Beispiel der Zustandsregelung deinen soll. 
Der besondere Vorteil der Regelungsnormalform besteht darin, dass die Werte $a_i$ die Koeffizienten des charakteristischen Polynoms 
\begin{equation}
det(\lambda\cdot \bs{I}-\bs{A}) = \prod^n_{i=1} (\lambda-\lambda_i) = \lambda^n+a_{n-1}\cdot \lambda^{n-1} + \hdots + a_1\cdot \lambda + a_0
\end{equation}
der Systemmatrix $\bs{A}$ sind. Wird der Regelkreis über den Reglervekor
\begin{equation}
\bs{k} = \begin{bmatrix}
k_1 & k_2 & \hdots & k_n
\end{bmatrix}
\end{equation}
geschlossen, ergibt sich sich die folgende Systemmatrix $\bs{A_g}$ des geschlossenen Regelkreises.
\begin{equation}
\begin{split}
\bs{A_g} = \bs{A}-\bs{b}\bs{k} &= \begin{bmatrix}
0 & 1 & 0 & \hdots & 0 \\
0 & 0 & 1 & \hdots & 0 \\
\vdots & \vdots & \vdots & \ddots & \vdots \\
-a_0 & -a_1 & -a_2 & \hdots & -a_{n-1}
\end{bmatrix}
-
\begin{bmatrix}
0 & 0 & 0 & \hdots & 0 \\
0 & 0 & 0 & \hdots & 0 \\
\vdots & \vdots & \vdots &  \ddots & \vdots \\
k_1 & k_2 & k_3 & \hdots & k_n
\end{bmatrix}
\\
&= \begin{bmatrix}
0 & 1 & 0 & \hdots & 0 \\
0 & 0 & 1 & \hdots & 0 \\
\vdots & \vdots & \vdots & \ddots & \vdots \\
-\overline{a}_0 & -\overline{a}_1 & -\overline{a}_2 & \hdots & -\overline{a}_{n-1}
\end{bmatrix}
\hspace{25pt} \vert \hspace{15pt} \overline{a}_i = a_i + k_{i+1}
\end{split}
\end{equation}
Die Systemmatrix $\bs{A_g}$ des geschlossenen Regelkreises liegt nun ebenfalls in Regelungsnormalform vor, weshalb wiederum für deren charakteristisches Polynom gilt:
\begin{equation}
det(\lambda\cdot \bs{I}-\bs{A_g}) = \prod^n_{i=1} (\lambda-\overline{\lambda}_i) = \lambda^n+\overline{a}_{n-1}\cdot \lambda^{n-1} + \hdots + \overline{a}_1\cdot \lambda + \overline{a}_0
\label{eq_ew_gRK}
\end{equation}
Werden nun die Eigenwerte $\overline{\lambda}_i$ des geschlossenen Regelkreises vorgegeben können durch (\cite{eq_ew_gRK}) die Koeffizienten $\overline{a}_i$ berechnet werden. Die Koeffizienten $a_i$ der Regelstrecke können aus der Matrix $\bs{A}$ abgelesen werden. Somit können aus der Beziehung
\begin{equation}
\overline{a}_i = a_i + k_i
\end{equation}
die einzelnen Regerfaktoren $k_i$ berechnet bzw. die folgende Vektorschreibweise verwendet werden.
\begin{equation}
\underbrace{
\begin{bmatrix}
\overline{a}_0 & \overline{a}_1 & \hdots &\overline{a}_{n-1}
\end{bmatrix}}_{\equiv \bs{\overline{a}}}
=
\underbrace{
\begin{bmatrix}
a_0 & a_1 & \hdots & a_{n-1}
\end{bmatrix}}_{\equiv \bs{a}}
+
\underbrace{
\begin{bmatrix}
k_1 & k_2 & \hdots & k_n
\end{bmatrix}}_{\equiv \bs{k}}
\end{equation}
\begin{equation}
\bs{k} = \bs{\overline{a}} - \bs{a}
\end{equation}
An diesem Entwurfsverfahren werden bereits einige interessante Eigenschaften der Zustandsregelung deutlich. Zunächst können die Eigenwerte des geschlossenen Regelkreises beliebig gewählt werden wodurch das dynamische Verhalten maßgeblich bestimmt wird. Des weiteren wird der Regler durch eine Matrix-Vektor-Multiplikation realisiert. Somit werden dem Regelkreis, im Gegensatz zu PID-Reglern, keine weiteren Pole durch die Reglerdynamik hinzugefügt. Zuletzt sei die einfache Berechnung der Reglermatrix erwähnt. Sowohl die Transformation auf Regelungsnormalform als auch die Ermittlung der Regerparameter sind numerische Operationen die mit Hilfe von Matlab durchgeführt werden können.

Nichtsdestotrotz wird das Entwurfsverfahren in dieser Arbeit nicht verwendet. Die Gründe hierfür sind, dass das primäre Ziel darin besteht eine robuste Regelung zur Stabilisierung des Systems zu entwerfen und nicht eine spezielle Dynamik des geschlossenen Regelkreises zu erzielen. Des weiteren können bei diesem Verfahren die Verläufe der physikalischen Größen $\varphi$, $u_K$ und $u_R$ nicht direkt beeinflusst werden. Durch die Vorgabe der Eigenwerte werden lediglich die Trajektorien der kanonischen Zustandsvariablen bestimmt, welche jeweils eine Linearkombination der ursprünglichen Zustände sind. Zuletzt fließt die Stellgröße $u$ bei dieser Vorgehensweise nicht in die Bestimmung der Reglermatrix ein. Ist die Stellgröße, wie in den hier behandelten Anwendungsfällen, begrenzt müssen diese Umstände bei dem Reglerentwurf unter Umständen beachtet werden.

\subsection{Linear quadratisch optimale Regelung}
Im vorherigen Abschnitt wurden die Nachteile der Eigewertvorgabe aufgezeigt, weshalb in hier der Entwurf eines linear quadratisch optimalen Reglers vorgestellt und an der realen Regelstrecke appliziert wird. Der Grundgedanke der optimalen Regelung ist es ein Regelgesetz zu entwerfen, welches ein zu einem geschlossenen Regelkreis führt, der im Sinne eines Gütekriteriums optimal ist. Hier wird das linear quadratische Gütekriterium
\begin{equation}
J = \sum^{\infty}_{k=0} [\bs{x}^T(k)\cdot \bs{Q} \cdot \bs{x}(k) + \bs{u}^T(k)\cdot \bs{R}\cdot \bs{u}(k)]
\end{equation}
verwendet. Die Matrizen $\bs{Q}$ und $\bs{R}$ dienen der Gewichtung des Zustands- bzw. Stellvektors und müssen positiv definit bzw. semidefinit sein. Durch die Gewichtungsmatrizen ist es möglich Forderung an den ursprünglichen Zustands- und Stellvektor direkt in den Reglerentwurf einfließen zu lassen. Der letztendliche Regler ergibt sich aus der Lösung des Optimierungsproblems, dass $J$ minimal wird. Nach [Ludyk, S.177] gilt für die Reglermatrix
\begin{equation}
\bs{K} = (\bs{R}+\bs{B}^T\bs{P}\bs{B})^{-1}\bs{B}^T\bs{P}\bs{A}\,,
\end{equation}
wobei $\bs{P}$ durch die Lösung der folgenden Matrix-Ricatti-Gleichung bestimmt wird.
\begin{equation}
\bs{P} = \bs{Q}+\bs{A}^T[\bs{P}-\bs{P}\bs{B}(\bs{R}+\bs{B}^T\bs{P}\bs{B})^{-1}\bs{B}^T\bs{P}]\bs{A}
\end{equation}
Ähnlich wie bei dem Entwurf durch Eigenwertvorgabe besteht ein Vorteil der linear quadratisch optimalen Regelung darin, dass das Optimierungsproblem mittels des Matlab-Befehls \textit{dlqr()} gelöst werden kann. Somit fällt der Rechenaufwand für die Ermittlung der Reglermatrix bei der Bewertung des Entwurfverfahrens nicht weiter ins Gewicht. Deshalb werden hier lediglich die Bedeutung der Gewichtsmatrizen und die Eigenschaften des resultierenden Regelkreises diskutiert.

Zunächst lässt sich zeigen [Lunze RT2, S.301], dass der geschlossene Regelkreis asymptotisch stabil ist insofern das System stabilisierbar ist. Bei einem stabilisierbaren System können alle instabilen Eigenvorgänge durch die Stellgröße und somit durch den Regler beeinflusst werden.\footnote{Die Systemeigenschaft heißt Steuerbarkeit und wird in dem anschließenden Abschnitt erläutert.} Hieraus folgt, dass alle Eigenwerte des geschlossenen Regelkreises im Einheitskreises liegen. Des weiteren lässt sich durch
\begin{equation}
\vert z\vert = e^{\sigma_i} \hspace{35pt} \vert \hspace{15pt} s_i = \sigma_i + \omega_i\cdot j
\end{equation}
der Abstand der Eigenwerte zum Ursprung der komplexen Ebene interpretieren. Je näher ein Eigenwert am Ursprung liegt, desto schneller wird die zugehörige Eigenbewegung. Im geschlossenen Regelkreis wird eine schnelle Systemdynamik durch hohe Stellgrößen verursacht, welche durch die Reglermatrix $\bs{K}$ bestimmt werden. Daraus kann geschlossen werden, dass die Erhöhung der Reglerfaktoren $k_{ij}$ die Eigenwerte des geschlossenen Regelkreises näher an den Ursprung rückt. Umgekehrt kann die Bedeutung der Gewichtungsmatrix $\bs{R}$ interpretiert werden. Je größer die Elemente von $\bs{R}$ werden, desto stärker fließt der Verlauf des Stellvektors in das Gütekriterium ein. Da dieses minimiert werden soll, resultieren kleinere Elemente der Reglermatrix und somit eine langsamere Systemdynamik.
Die Matrix $\bs{Q}$ wirkt entgegengesetzt. Das das Gütekriterium den zeitlichen Verlauf des Zustandvektors erfasst, führt eine Erhöhung der Matrix $\bs{Q}$ auf eine schnellere Eigenbewegung, welche durch erhöhte Reglerwerte erreicht wird. Hier zeigt sich ein Vorteil der linear quadratisch optimalen Regelung. Bei der Eigenwertvorgabe kann der Verlauf einer Zustandsgröße nur indirekt durch die Definition der Eigenbewegung beeinflusst werden. Im Gegensatz dazu erfolgt durch die Diagnoalelemente der Matrix $\bs{Q}$ eine direkte Gewichtung der Zustandsgrößen.\footnote{Allerdings sei angemerkt, dass sich durch die Gewichtungsmatrix $\bs{Q}$ keine Entkopplung der Zustandsgrößen erzwingen lässt. Für die Entkopplung der Zustandsgrößen müssen nicht nur die Eigenwerte sondern auch die Eigenvektoren durch den Regler beeinflusst werden. Dies ist nur möglich wenn die Reglermatrix mehr Elemente als zu regelnde Zustandsgrößen besitzt [Lunze RT2, S.254 ff]. Des weiteren sind solche System in der Realität oftmals nicht realisierbar, wären beispielsweise alle Eigenwerte und -vektoren des Würfels frei wählbar könnten die Größen $\varphi$ und $\dot{\varphi}$ voneinander entkoppelt werden.}
Da das Zeitverhalten des geschlossenen Regelkreises in dem hier behandelten Anwendungsfall eine untergeordnete Rolle spielt, wird nun die Empfindlichkeit gegenüber Störungen betrachtet. Aus dem Reglergesetzt
\begin{equation}
\bs{u} = -\bs{K}\cdot\bs{x}
\end{equation}
geht hervor, dass sich ein beliebiges Messrauschen proportional zur Reglermatrix in der Stellgröße widerspiegelt. Folglich beeinflusst das Entwurfsverfahren nicht direkt die Störempfindlichkeit des Regelkreises. Vielmehr wird diese durch die Lage der Eigenwerte bestimmt. Je näher diese an dem Einheitskreis liegen, desto robuster wird das System gegenüber Messrauschen.
Ein weiterer Vorteil der linear quadratisch optimalen Regelung besteht in derer Robustheit gegenüber Parameteränderungen. Diese Eigenschaft beschreibt wir groß die Abweichungen einzelner Parameter sein können um nach wie vor einen stabilen Regelkreis zu erhalten. In [Lunze RT2, S. 303ff] werden diese Eigenschaften mit Hilfe des Phasen- und Amplitudenrandes erläutert. Ludynk ergänzen
