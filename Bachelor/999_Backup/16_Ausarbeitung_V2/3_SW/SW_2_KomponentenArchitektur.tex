\section{Komponentenarchitektur für mechatronische Anwendungen}
In dem letzten Abschnitt wurden die elementaren Funktionen des Regelkreises, wie die Peripherieinteraktion und der Signalfluss, implementiert. Um eine effiziente Versuchsdurchführung zu ermöglichen müssen allerdings weitere Funktionalitäten bereitstehen. Einerseits muss einzelne Elemente des Regelkreises während der Ausführung konfigurierbar sein. Beispielsweise soll zwischen unterschiedlichen Regler umgeschaltet werden und einzelne Parameter geändert werden können. Des weiteren müssen die gemessen und berechneten Werte des Signalflusses an einen Entwicklungsrechner übertragen werden. Dort werden die Daten visualisiert und für eine spätere Analyse gespeichert.
Folglich muss ein Kommunikationskonzept implementiert werden um Daten zwischen der Ziel- und Entwicklungsplattform auszutauschen. Um die Berechnung des Regelkreises und die Kommunikationsaufgaben voneinander zu trennen wird eine Komponentenarchitektur eingeführt. Die erste Komponente, welche als Regelungskomponente bezeichnet wird, führt die Berechnung des Signalflusses und die Interaktion mit den Peripheriegeräten durch. Des weiteren übernimmt sie die logische Steuerung der Versuchsabläufe. Die Kommunikationskomponente ist für die Verbindung mit dem Entwicklungsrechner verantwortlich und ermöglicht den Datenaustausch zwischen den beiden Plattform. Hierfür wird ein WebSocketServer verwendet (siehe Abschnitt...).
\begin{figure}[!h]
\centering
\includegraphics[width=\linewidth]{img/SW_1_KA_BSB.pdf}
\caption{Blockschaltbild Gesamtsystem, Quelle: eigene Darstellung}
\end{figure}
Durch die Verwendung der Komponentenarchitektur enstehen einige Vorteile. Zunächst können die beiden Komponenten in unterschiedlichen Threads ausgeführt werden. Dadurch wird die Bearbeitung der beiden Aufgabenbereiche zum Großteil entkoppelt. Der Datenaustausch zwischen den Komponenten reduziert sich auf eine kontrollierte Schnittstelle, wodurch die Fehleranfälligkeit des Gesamtsystems minimiert wird. Des weiteren können die Aufgaben priorisiert werden, da der Scheduler eine preemptive Round-Robin-Strategie verfolgt. Somit kann die Ausführung der Regelungskomponente, welche für die Bearbeitung der zeit- und sicherheitsrelevanten Aufgaben zuständig ist, gegenüber der Kommunikationseinheit priorisiert werden. Hieraus resultiert, dass das Zeitverhalten der Regelung als deterministisch angenommen werden kann. 

Für die Implementierung wird das Interface \textit{IRunnable} definiert, welches virtuelle Methoden zur Initialisierung und Ausführung der Komponenten vorschreibt. Diese Schnittstelle wird von der abstrakten Klasse \textit{AComponentBase} geerbt, welche Membervariablen zum Datenaustausch besitzt. Die letzendlichen Komponenten werden in Form der beiden Klassen \textit{CControlComp} und \textit{CCommComp} realisiert. Die Erzeugung der Threads erfolgt mit Hilfe der Klasse \textit{CThread}, deren Instanz als Trägerobjekte der Threads agieren.
\begin{figure}[!h]
\centering
\includegraphics[width=0.6\linewidth]{img/SW_1_KA_KD.pdf}
\caption{Klassendiagramm der Komponenten, Quelle: eigene Darstellung}
\end{figure}

Im nächsten Schritt muss ein Weg zum Datenaustausch zwischen den Komponenten etabliert werden. Einerseits müssen Messdaten und Signale aus dem Regelkreis von der Regelungs- and die Kommunikationseinheit gesendet werden, welche diese anschließend an den Host-PC weiterleitet. Andererseits werden die Steuerbefehle des Host-PCs von der Kommunikationskomponente empfangen und an die Regelungskomponente gereicht. Da in diesem Anwendungsfall lediglich Datenpakete versendet werden, die kleiner als 32 Byte sind, werden Nachrichten für deren Austausch verwendet. Die Nachrichten werden in Form der Klasse \textit{CMessage} implementiert, welche aus einem Datenfeld, einem Ereignis und einem Zeitstempel komponiert werden. Das Eregnis wird als Enumeration realisiert und gibt den Inhalt der Daten bzw. den jeweiligen Befehl wieder. Der Zeitstempel gibt den Abtastzeitpunkt der Signale aus dem Regelkreis wieder. 
\begin{lstlisting}
enum class EEvent
{
	DEFAULT_IGNORE = 0,
	ASDF = 1
}
class CMessage
{
public:
	EEvent getEvent() const;
	UInt8* getDataPtr();
	...
public:

private:
	static constexpr UInt32 sDataSize = 32U;

	EEvent  mEvent;
	Float32 mTime;	
	UInt8   mData[sDataSize];
};
\end{lstlisting}
Des weiteren besitzt die Klasse Methoden und Konstruktoren um die Datenfelder entsprechend zu füllen und auszulesen. Um die Nachrichten zu empfangen besitzen die Komponenten Eingangspuffer die als Queues implementiert werden. Die Erzeugung der Nachrichten wird von einem Proxy übernommen, der Methoden für die unterschiedlichen Ereignisse bereitstellt. Des weiteren kennt der Proxy die Queues der Komponenten und legt neue Nachrichten, in Abhängigkeit von dem jeweiligen Event, in die Eingangspuffers des zugehörigen Empfängers.
\begin{lstlisting}
class CProxy
{
	public:
		bool transmitStateVector(const StateVector& x);
		...
};
\end{lstlisting}
Mit Hilfe der Nachrichtenkommunikation kann auch das Ablaufschema der Komponenten verallgemeinert werden. Beim Starten der Threads werden die Komponenten zunächst über den Aufruf ihrer \textit{init()}-Methoden initialisiert, anschließend wird die \textit{run()}-Methode ausgeführt, welche in diesem Fall eine Endlosschleife darstellt. In einem Durchlauf wird geprüft ob neue Nachrichten vorhanden sind, ist dies der Fall wird die Nachricht über die \textit{dispatch()}-Methode verarbeitet. Andernfalls legt sich der Thread schlafen bis neue Nachrichten zur Verfügung stehen. Hieraus ergeben sich zwei Vorteile. Einerseits werden die Komponenten nach einem einheitlichen Konzept entworfen, lediglich die Implementierung der \textit{init()}- und \textit{dispatch()}-Methode unterscheiden sich, andererseits wird durch die Synchronisation der Komponenten deren Laufzeit reduziert.