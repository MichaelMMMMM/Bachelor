\section{Aufbau der Desktop-Anwendung}
Der letzte Teil der Software-Infrastruktur besteht in einer Anwendung, welche in einer gewöhnlichen Desktopumgebung ausgeführt wird und die Interaktion mit der Zielplattform ermöglicht. In diesem Fall wird als Desktopbetriebssystem Ubuntu 16.04 verwendet. Die Anwendung stellt eine graphische Benutzeroberfläche zur Verfügung, welche einerseits relevante Daten der Versuche visualisiert und andererseits Bedienelemente bietet, um Versuchskonfigurationen vorzunehmen.

Für die Implementierung der Benutzeroberfläche wird die Bibliothek Qt verwendet, welche übliche Bedien- und Grafikelemente zur Verfügung stellt. Der Aufbau der Anwendung ist in zwei Threads unterteilt. Der Erste verwaltet eine Instanz der Klasse \textit{CClient}. Dieses Objekt stellt das Gegenstück zu dem Server dar, der auf dem BeagleBone Black ausgeführt wird. Die Klasse \textit{CClient} bietet ebenfalls die Methoden \textit{receiveMessage()} und \textit{transmitMessage()}, um Nachrichten zu Empfangen bzw. zu Versenden. Zudem kann die Methode \textit{connectToServer()} genutzt werden, um eine Verbindung mit dem BeagleBone Black herzustellen.

Der zweite Thread verwaltet die Benutzeroberfläche. Diese setzt sich aus Plots zur Darstellung der Versuchsdaten und den nötigen Bedienelementen für die Versuchskonfiguration zusammen. Für die Kommunikation zwischen den beiden Threads wird Qts Signal/Slot-Prinzip verwendet. Der Qt-Meta-Compiler ermöglicht es Methoden als Signale bzw. Slots zu deklarieren. In der Anwendung können die Signale eines Objekts mit den Slots eines weiteren Objekts verbunden werden. Anschließend kann ein Objekt seine Signale emittieren, woraufhin der verbundene Slot aufgerufen wird. Als Beispiel sei eine Gruppe von Schalter zur Auswahl des Reglerkonzeptes genannt. Wird einer der Schalter gedrückt, löst dieser ein Signal aus, welches mit einem Slot des Grafik-Threads verbunden ist. In diesem wird die Auswahl ausgewertet und ein entsprechendes Signal ausgelöst. Das Signal des Grafik-Threads ist wiederum an einen Slot des Kommunikations-Threads gekoppelt. In dem Slot wird eine Instanz von \textit{CMessage} mit dem passenden Event angelegt und über den Client an das BeagleBoneBlack gesendet. Umgekehrt wird bei dem Eintreffen von Versuchsdaten ein entsprechendes Signal des Kommunikations-Thread emittiert. In dem zugehörigen Slot des Grafik-Threads werden die Daten verarbeitet und in den Plots angezeigt.