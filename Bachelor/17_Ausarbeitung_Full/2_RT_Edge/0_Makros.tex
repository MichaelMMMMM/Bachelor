\def\presuper#1#2%
	{\mathop{}%
	\mathopen{\vphantom{#2}}^{#1}%
	\kern-\scriptspace%
	#2}
%Display vecotr in a reference frame
\newcommand{\vecBS}[4]{\presuper{#1}{\begin{pmatrix}
#2 \\ #3 \\ #4
\end{pmatrix}}}
%Boldsymbol shortcut
\newcommand{\bs}[1]{\boldsymbol{#1}}
%Bezugssystemdefinition
\newcommand{\defBS}[1]{\{#1\} [ \bs{e}_{{#1}_1},\bs{e}_{{#1}_2}, \bs{e}_{{#1}_3} ]}
%Projektionsmatrix
\newcommand{\pMat}[2]{\presuper{#1}{\bs{P}}^{#2}}
%Differenation in Respekt zu BS
\newcommand{\diffIn}[3]{\frac{\presuper{#1}{d{#2}}}{d#3}}
\newcommand{\partialDiffIn}[3]{\frac{\presuper{#1}{\partial{#2}}}{\partial #3}}
%Geschwindigkeit/Beschleunigung
\newcommand{\vel}[3]{\presuper{#1}{\bs{#2}}^{#3}}

%Rightarrow with spaceing
\newcommand{\rArrow}{\hspace{5pt}\rightarrow\hspace{5pt}}
%Inneres Produkt
\newcommand{\inProd}[2]{\langle {#1}, {#2} \rangle}

%System macro
\newcommand{\cSS}[3]{\textfrak{S}($\bs{#1}$,$\bs{#2}$,$\bs{#3}$)}
\newcommand{\dSS}[3]{\textfrak{D}($\bs{#1}$,$\bs{#2}$,$\bs{#3}$)}

%Laplace transform sign with spaces
\newcommand{\myLaplace}{\hspace{15pt}\laplace\hspace{15pt}}

\newcommand*{\signed}[1]{%
        \nolinebreak[3]\hspace*{\fill}\mbox{\emph{#1}}
    }
%Vektor norm
\newcommand{\norm}[1]{\left\lVert#1\right\rVert}
%Kreuzproduktmatrix
\newcommand{\kProdMat}[1]{\left[ #1 \right]_\times}

%Nicht kursiver Subskript
\newcommand{\idx}[1]{_{\text{#1}}}