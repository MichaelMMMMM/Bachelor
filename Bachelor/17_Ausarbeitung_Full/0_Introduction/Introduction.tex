\chapter{Einleitung}
Im Sommersemester 2016 begann im Rahmen eines Entwicklungsprojektes an der Hoschule Karlsruhe die Entwicklung eines balancierend Würfels. Die Idee hierfür stammt von dem Cubli, welcher an der ETH Zürich entwickelt wurde. Die Aufgabe des Projekts besteht darin einen Würfel mit einer Kantenlänge von $15cm$ zu entwerfen, der in der Lage ist auf seinen Kanten und Ecken zu balancieren. Hierfür sind im Inneren des Würfels drei Motoren montiert an welchen jeweils eine Schwungmasse angebracht ist. Die Drehmomente der Motoren dienen als Stellgrößen eines Reglers um den Würfel in einer aufrechten Position zu balancieren. Des weiteren können die Schwungmassen genutzt werden um den Würfel aus einer beliebigen Position auf seine Kanten bzw. Ecken aufzurichten. Hierfür werden die Schwungmassen zunächst beschleunigt und anschließend mittels einer mechanischen Bremse gestoppt. Dadurch wird deren Drehimpuls auf das Würfelgehäuse übertragt.

In dem vorangehenden Entwicklungsprojekt wurde zunächst eine einzelne Würfelseite mit einem Motor konzipiert. Dieses Modell lässt sich mit dem auf einer Kante balancierenden Würfel vergleichen. Die Bewegungsgleichungen der Würfelseite wurden mit Hilfe des Lagrange-Formalismus hergeleitet und anschließend für den Entwurf eines Zustandsreglers verwendet. Der Regler ermöglicht das Balancieren der Würfelseite in der aufrichten Position.