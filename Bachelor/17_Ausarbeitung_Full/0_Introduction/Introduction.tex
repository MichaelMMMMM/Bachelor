\chapter{Einleitung}
Im Sommersemester 2016 begann im Rahmen eines Entwicklungsprojektes an der Hoschule Karlsruhe die Entwicklung eines balancierend Würfels. Die Idee hierfür stammt von dem Cubli, welcher an der ETH Zürich entwickelt wurde. Die Aufgabe des Projekts besteht darin einen Würfel mit einer Kantenlänge von $15cm$ zu entwerfen, der in der Lage ist auf seinen Kanten und Ecken zu balancieren. Hierfür sind im Inneren des Würfels drei Motoren montiert an welchen jeweils eine Schwungmasse angebracht ist. Die Drehmomente der Motoren dienen als Stellgrößen eines Reglers um den Würfel in einer aufrechten Position zu balancieren. Allerdings ist es nicht möglich mit Hilfe der Motormomente den Würfel aus einer beliebigen Position auf seine Kanten oder Ecken zu richten. Der Grund hierfür ist, dass das maximale Motormoment nur in einem begrenzten Gebiet das Gravitationsmoment überwinden kann. Um dieses Problem zu Lösen werden die Schwungmassen als Energiespeicher verwendet. Die Schwungmasse werden zunächst beschleunigt und im Anschluss mittels einer mechanischen Bremse gestoppt. Dadurch wird deren Drehimpuls auf das Würfelgehäuse übertragt und es entsteht ein genügendes Drehmoment um den Würfel aufzurichten.

In dem vorangehenden Entwicklungsprojekt wurde zunächst eine einzelne Würfelseite mit einem Motor konzipiert. Dieses Modell lässt sich mit dem auf einer Kante balancierenden Würfel vergleichen. Die Bewegungsgleichungen der Würfelseite wurden mit Hilfe des Lagrange-Formalismus hergeleitet und anschließend für den Entwurf eines Zustandsreglers verwendet. Der Regler ermöglicht das Balancieren der Würfelseite in der aufrichten Position. Um die Zustandsgrößen zu messen sind an der Würfelseite zwei Sensormodule montiert, welche deren Beschleunigung- und Winkelgeschwindigkeit in Richtung von drei Achsen erfassen. Hierbei werden die Beschleunigungen genutzt um nach [Cubli1] die Ausrichtung der Würfelseite zu bestimmen. Des weiteren sind in dem Motor Hall-Sensoren integriert um die Winkelgeschwindigkeit der Schwungmasse zu ermitteln. Für die Umsetzung des zeitdiskreten Reglers wird ein \ac{BBB} verwendet, auf welchen ein Linux-Betriebssystem ausgeführt wird. Der letzte Aufgabenteil des Entwicklungsprojektes bestand in der Konstruktion und Fertigung des Würfels.

Diese Arbeit beschäftigt sich nun mit der Entwicklung und Implementierung eines Reglers um den Würfel auf einer Ecke zu balancieren. Den Ausgangspunkt des Reglerentwurfs stellt ein Modell dar, welches aus den Bewegungsgleichungen des Würfels hervorgeht. Da sich in der Vorarbeit der Lagrange-Formalismus als ineffiziente Vorgehensweise zur Bestimmung der Bewegungsgleichungen erwiesen hat, wird in dieser Arbeit Kanes Methodik [Kane] verwendet.

Des weiteren stellt das Balancieren auf einer Ecke im Vergleich zu einer Kante ein deutlich komplexeres Regelungsproblem dar. Die Lösung dieses Problems erfordert ein Verständnis der Zustandsregelung zugrunde liegenden Systemtheorie. Aus diesen Gründen wird in dieser Arbeit zunächst das Balancieren auf einer Kante untersucht. An diesem simplen Beispiel werden mittels Kanes Methodik die Bewegungsgleichungen hergeleitet. Anschließend werden die theoretischen Grundlagen der Zustandsregelung diskutiert und ein LQR-basierter Regler entwickelt. Dieser wird danach an der Regelstrecke validiert.

Im nächsten Schritt werden die Bewegungsgleichung für das Balancieren auf einer Ecke ermittelt. Die Bewegungsgleichungen werden dann in eine Zustandsraumdarstellung überführt, welche für den Entwurf des Reglers genutzt wird. Hier wird zunächst die Beobachtbar- und Steuerbarkeit des Würfels diskutiert und an Hand einer Simulation untersucht. Des weiteren wird die Direktionalitätsproblematik erläutert, welche durch die Stelgrößenbeschränkung von Mehrgrößensystemen entsteht. Unter Beachtung dieser Probleme wird mit Hilfe des LQR-Verfahren ein Zustandsregler entworfen, der an der Regelstrecke erprobt wird. Hier zeigt sich, dass der Regler die Forderung an den geschlossenen Kreis nicht vollständig erfüllt. Deshalb wird zuletzt eine empirische Optimierung an Hand der Versuchsergebnisse durchgeführt.

Der nächste Teil der Arbeit beschäftigt sich mit der Erfassung der Zustandsgrößen. Hierfür sind an dem Würfel sechs Sensormodule angebracht, welche jeweils einen Beschleunigungs- und Drehratensensor besitzen. Die Messung der Winkelgeschwindigkeiten des Würfels und der Schwungmassen wird dabei nur kurz behandelt, da das Konzept aus der Vorarbeit übernommen werden kann. Allerdings muss das Verfahren zur Bestimmung der Winkel, welche die Ausrichtung des Würfels beschreiben, auf den Fall der räumlichen Bewegung erweitert werden. Des weiteren wird ein Komplementärfilter implementiert um die Störgrößen, welche auf die Sensoren wirken, zu eliminieren. Im letzten Schritt wird für das Balancieren auf einer Kante ein Beobachter entworfen. Dieser ist in der Lage die Ausrichtung des Würfels zu schätzen. Dadurch ist es möglich die Anzahl der nötigen Sensoren zu reduzieren und Auswirkung der Störgrößen zu minimieren.

Der letzte Teil der Arbeit behandelt die softwareseitige Implementierung der Regler. Hier wird zunächst die Kombination des BeagleBone Black mit einem Linux-Betriebssystem als Zielplattform diskutiert. Anschließend wird die Ansteuerung und Auswertung der Peripherieeinheiten erläutert. Im nächsten Schritt wird eine Komponenten-Architektur vorgestellt, welche sich auf weitere mechatronische Anwendungen übertragen lässt. Ziel hierbei ist es eine effiziente Infrastruktur für die Durchgeführung von Versuchen zu schaffen. Hierfür werden Ansätze der Template-Meta-Programmierung verwendet. Diese ermöglichen es Algorithmen der Signalverarbeitung und Regelungstechnik unter minimalem Aufwand zu beliebigen Signalflüssen zusammenzusetzen. Des weiteren wird eine Verbindung zu einem Entwicklungsrechner implementiert, welche auf dem TCP/IP-Protokoll basiert. Dadurch können Messdaten und Signale aufgezeichnet und auf einer Benutzeroberfläche visualisiert werden.
