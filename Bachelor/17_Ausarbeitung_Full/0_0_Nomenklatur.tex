%==============================================================
%
% Nomenklatur
%
%==============================================================
\chapter*{Nomenklatur}
  \label{Nomenklatur}%
   % Die folgende Zeile erzwingt einen Eintag ins Inhaltsverz.
  \addcontentsline{toc}{chapter}{Nomenklatur}%
%-----------
\manualmark
\markright{Nomenklatur}
\markleft{Nomenklatur}
%\ohead[]{\headmark}
\begin{longtable}{p{0.15\textwidth} p{0.75\textwidth}}  % ACHTUNG Breite 0.9
%-----------------------------------------------
\multicolumn{2}{l}{%
\textbf{\textsf{\large Schreibeweise mathematischer Symbole}}
}\\
$\bs{a}$ & Vektor wird fett und klein geschrieben \\
$\bs{a}^T$ & Transponierter von Vektor $\bs{a}$ \\
$\bs{A}$ & Matrix wird fett und groß geschrieben \\
$\bs{A}^T$ & Transponierte der Matrix $\bs{A}$ \\
$\inProd{\bs{a}}{\bs{b}}$ & Skalarprodukt der Vektoren $\bs{a}$ und $\bs{b}$ \\
$\kProdMat{\bs{b}}$ & Kreuzproduktmatrix des Vektors $\bs{b}$ ($\bs{b}\times \bs{a}$ = $\kProdMat{\bs{b}}\cdot \bs{a})$ \\
                      &  \\ % LEERZEILE
\multicolumn{2}{l}{%
\textbf{\textsf{\large Symbole der technischen Mechanik}}
}\\
$\presuper{A}{\bs{b}}$ & Vektor $\bs{b}$ aus Perspektive des Bezugssystem $A$ \\
$\vel{A}{v}{B}$ & Geschwindigkeit von $B$ relativ zu $A$ \\
$\vel{A}{a}{B}$ & Beschleunigung von $B$ relativ zu $A$ \\
$\vel{A}{\omega}{B}$ & Winkelgeschwindigkeit von $B$ relativ zu $A$ \\
$\vel{A}{\alpha}{B}$ & Winkelbeschleunigung von $B$ relativ zu $A$ \\\
$q$ & Generalisierte Koordinate \\
$u$ & Generalisierte Geschwindigkeit \\
$\vel{A}{v}{B}_i$ & Partielle Geschwindigkeit $i$ von $B$ relativ zu $A$ \\
$\vel{A}{\omega}{B}_i$ & Partielle Winkelgeschwindigkeit $i$ von $B$ relativ zu $A$ \\
$F$ & Generalisierte aktive Kraft \\
$F^*$ & Generalisierte Trägheitskraft \\
$\bs{T}^{A/B}$ & Auf $A$ wirkendes Drehmoment mit dem Drehpunkt $B$ \\
$\bs{T}^{A/B}_*$ & Auf $A$ wirkendes Trägheitsmoment mit dem Drehpunkt $B$ \\
$I$ & Trägheitsskalar bzw. Massenträgheitsmoment \\
$\bs{I}$ & Trägheitstensor \\
\newpage
\multicolumn{2}{l}{%
\textbf{\textsf{\large Symbole der Regelungstechnik}}
}\\
\textfrak{S} & Zeitkontinuierliches System \\
\textfrak{D} & Zeitdiskretes System \\
$\bs{x}$	 & Zustandsvektor \\
$\bs{u}$	 & Eingangsvektor \\
$\bs{y}$     & Ausgangsvektor \\
$\bs{A}$     & Systemmatrix \\
$\bs{B}$	 & Eingangsmatrix \\
$\bs{C}$	 & Ausgangsmatrix \\
$\bs{K}$	 & Reglermatrix \\
$\bs{L}$	 & Beobachtermatrix \\
$\bs{0}^{i\times j}$ & Nullmatrix des Raumes $\mathbb{R}^{i\times j}$ \\
$\bs{I}^{i\times i}$ & Einheitsmatrix des Raumes $\mathbb{R}^{i\times i}$ \\
$s_{\varphi}$ &	 Sinus des Winkels $\varphi$ \\
$c_{\varphi}$ &  Cosinus des Winkels $\varphi$ \\
\\
\multicolumn{2}{l}{%
\textbf{\textsf{\large Abkürzungsverzeichnis}}
}\\
SISO 		& Single-Input-Single-Output \\
MIMO		& Multiple-Input-Multiple-Output \\
SPI			& Serial Peripheral Interface \\
CS			& Chip-Select \\
ADC			& Analog-to-Digital Converter \\
GPIO		& General Purpose Input/Output \\
BBB			& BeagleBone Black \\
FSM			& Finite-State-Machine \\
LQR			& Linear-Quadratische Regelung \\


\end{longtable}

\cleardoublepage




















