%==============================================================
%
% Nomenklatur
%
%==============================================================
\chapter*{Nomenklatur%
  \footnote{ das ist hier nur beispielhaft; es werden sonst nur (alle) tatsächlich genutzte Zeichen aufgeführt!\\
	beim Sortieren: erst kleine, dann große Buchstaben.} }%
  \label{Nomenklatur}%
   % Die folgende Zeile erzwingt einen Eintag ins Inhaltsverz.
  \addcontentsline{toc}{chapter}{Nomenklatur}%
%-----------
\manualmark
\markright{Nomenklatur}
\markleft{Nomenklatur}
%\ohead[]{\headmark}
%\begin{longtable}{p{3cm} p{2cm} p{\textwidth}}  % ACHTUNG Breite
\begin{longtable}{p{0.15\textwidth} p{0.15\textwidth} p{0.6\textwidth}}  % ACHTUNG Breite 0.9
%-----------------------------------------------
\multicolumn{3}{l}{%
\textbf{\textsf{\large Lateinische Formelzeichen}}
}\\
%-----------------------------------------------
%
%----Große Buchstaben----
&&\\
%----kleine Buchstaben----
$g_i$                     & {m/s$^2$}       & Vektor der Schwerkraft    \\
$k$                       & {m$^2$/s$^2$}   & turbulente kinetische Energie \\
$l$                       &   m                & Längenmaß \\
$\dot m$                  &   kg/s             & Massenstrom \\
$p$                       &   N/m$^2$          & Druck \\
$R$                      &   -                 &  Residuenmatrix   \\
{\textbf{R}}             & kJ/(kmol K)         & Universelle Gaskonstante \\
$s$ 										& 	{J/(kg~K)}         & spezifische Entropie \\
$ S $                    & J/K                 & Entropie \\
$S_{ij}$                 &   1/s               &  Scherrate   \\
$t$                       &   s                & Zeit \\
$ T $                    & K                   & Temperatur \\
$u,v,w$                   &   m/s              & Geschwindigkeitskomponente \\
                      & & \\ % LEERZEILE

%----Griechische Buchstaben-----------------------------------------------------------------------
\\
\multicolumn{3}{l}{%
\textbf{\textsf{\large Griechische Formelzeichen}}
 }\\
$\alpha$                &    -                & allgemeine Zustandsgröße \\
$\beta^*$               &    -                & Konstante des Turbulenzmodells \\
$\xi$                   &    kg/kg            & Massenanteil\\
$\sigma_i$              &    -                & Konstante des Turbulenzmodells \\
$\tau_{ij}$             &    N/m$^2$          & Spannungstensor   \\
$\omega$                &    1/s              & Frequenz der turbulenten Schwankung   \\
$\Phi$                  &    -                & allgemeine Konstante des Turbulenzmodells   \\

 & & \\ 
%--------------------------------------------------------------------------
\\
\multicolumn{3}{l}{%
\textbf{\textsf{\large Indizes}}
}\\
%-----------------------------------------------
bulk                   & & mittlere Geschwindigkeit \\ 
$i$                    & & Richtungsindex, Spezies \\ 
$j$                    & & Summationsindex, Element \\ 
$n$                    & & Zeitschritt \\
t                      & & turbulent, total, tangential \\
U                      & & Umgebung \\
                       & & \\ %LEERZEILE
%--------------------------------------------------------------------------
\newpage
\multicolumn{3}{l}{%
\textbf{\textsf{\large Besondere Zeichen}}
}\\
%---------------------------------------------
$\mathrm{d}$        & &      steiles d: vollständiges Differenzial\\  
$\partial$   & &      partieller Differentialoperator\\
$\Delta$     & &      Differenz\\
 $:=$         & &      definiert durch\\
$\equiv$     & &      identisch\\
$\propto$    & &      proportional\\
$\approx$    & &      etwa \\
(1.1)        & &      Gleichungsnummer, die erste Zahl gibt die\\
             &&         Nummer des Kapitels an, die zweite Zahl ist fortlaufend im Kapitel\\
$[12]$       & &      Nummer im Quellenverzeichnis\\             

 & & \\ % LEERZEILE

%--------------------------------------------------------------------------
\\
\multicolumn{3}{l}{%
\textbf{\textsf{\large Dimensionslose Kennzahlen}}
}\\
%---------------------------------------------
$\mathit{Re} := w L /\nu$  & &   Reynolds-Zahl\\
 & & \\ % LEERZEILE

%---------------------------------------------
\\
\multicolumn{3}{l}{%
\textbf{\textsf{\large Abkürzungen\footnote{Abkürzungen, die bereits im Duden stehen, werden nicht aufgeführt. }}}
}\\
%---------------------------------------------

HsKA     & & Hochschule Karlsruhe -- Technik und Wirtschaft\\
IKKU     & & Institut für Kälte-, Klima- und Umwelttechnik\\
IMP      & & Institute of Materials and Processes\\
MMT      & & Maschinenbau und Mechatronik \\
PDF      & & Portable Document Format \\ 
RKS      & & Redlich-Kwong-Soave \\
%SST      & & Shear Stress Transport \\
%TKE      & & Turbulente Kinetische Energie \\


\end{longtable}

\cleardoublepage




















