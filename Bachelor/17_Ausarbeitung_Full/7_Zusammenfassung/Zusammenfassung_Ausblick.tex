\chapter{Zusammenfassung und Ausblick}
In dieser Arbeit wurde ein Konzept entwickelt um einen Würfel auf einer seiner Ecken zu balancieren. Die hierfür nötigen Aufgaben lassen sich in zwei Gebiete unterteilen. Auf der einen Seite steht die analytische Entwicklung des Regelungskonzeptes. Hierfür wurden zunächst mittels Kanes Methodik die Bewegungsgleichungen des Systems hergeleitet. Im nächsten Schritt wurden diese in eine diskrete Zustandsraumdarstellung überführt. An Hand dieses Modells wurde ein Mehrgrößenregler entworfen, wobei das LQR-Verfahren verwendet wurde. Anschließend wurde die Bedeutung der nicht steuer- und beobachtbaren Unterräume des Systems analysiert. Im Anschluss wurde der Regler an dem Würfle erprobt, verbessert und dessen Stabilität bewiesen.
Die letzte Teilaufgabe bestand in der Erfassung der Zustandsgrößen mit Hilfe der vorhandenen Sensorik. Hierbei stellt die die Bestimmung der Winkel $\varphi_i$, welche die Ausrichtung des Würfels beschreiben, ein besonderes Problem dar, da diese lediglich an Hand des Erdbeschleunigungvektors geschätzt werden können. Als Lösung wurde ein Algorithmus vorgestellt, der es ermöglicht die resultierenden Beschleunigungssignale aus den Messwerten der Beschleunigungssensoren zu eliminieren. Des weiteren wurde ein Komplementärfilter implementiert um die Güte der Winkelsignale zu erhöhen.

Der zweite Aufgabenteil besteht in der Implementierung des Regelungskonzeptes. Hierbei wurde das Ziel verfolgt eine Infrastruktur zu schaffen, welche es ermöglicht während des Entwicklungsprozess Versuche effizient zu implementieren und durchzuführen. Hierfür wird ein Ansatz der Template-Meta-Programmierung verfolgt, welcher den Implementierungsaufwand von Signalflüssen minimiert. Im Anschluss wurde eine Komponentenarchitektur entworfen um die Hauptaufgaben der Anwendung voneinander zu trennen. Hieraus ergibt sich der Vorteil, dass die Teile der Anwendung, welche für die Regelungstechnik relevant sind, priorisiert werden können und somit ein nahezu deterministisches Zeitverhalten resultiert. Des weiteren entsteht durch die Komponentenarchitektur eine übersichtliche Programmstruktur, die für beliebige mechatronische Anwendungen wiederverwendet werden kann. Hierbei muss lediglich der Kontroll- und Signalfluss der Versuche angepasst werden, was sich mit Hilfe der hier entwickelten Template-Methoden effizient realisieren lässt. Im letzten Schritt wurde eine Desktopapplikation entworfen um mit der Zielplattform zu interagieren. Die Anwendung besteht aus einer graphischen Benutzeroberfläche, die einerseits relevante Versuchsdaten visualisiert und andererseits Bedienelemente bietet um den Versuchsablauf zu konfigurieren. Für die Implementierung der Anwendung wurde die Bibliothek Qt verwendet.

An dieser Stelle sind weitere Verbesserung möglich. Ein Ansatz besteht darin den TCP/IP-Server durch einen Webserver zu ersetzen. In dieser Konfiguration wird die Desktopanwendung als Website implementiert und mittels eines Browsers ausgeführt. Die Vorteile dieser Vorgehensweise bestehen darin, dass der Implementierung der Anwendung weniger Zeit in Anspruch nimmt. Des weiteren handelt es sich um eine plattformunabhängige Lösung. Zum Beispiel kann die Anwendung mit Hilfe eines Browsers auch auf einem Smartphone oder Tablet ausgeführt werden. Dieser webbasierte Ansatz wurde im Rahmen dieser Arbeit ebenfalls verfolgt. Allerdings ergab sich das Problem, dass die verwendete JavaScript-Bibliothek Flot die erforderlichen Datenmengen nicht in einer ausreichenden Geschwindigkeit darstellen kann. Somit ist zu untersuchen, welche Alternativen für die Visualisierung bestehen.

Auf der Seite der Modellbildung und Regelungstechnik sind ebenfalls Verbesserung möglich. Zunächst ist eine Systemidentifikation durchzuführen um die Modellgüte zu bewerten und zu verbessern. Der Grund, weshalb dies nicht im Rahmen dieser Arbeit erfolgt ist, besteht darin, dass hier lediglich diskretisierte Modelle der kontinuierlichen Regelstrecken verwendet werden. Beispielsweise ist es möglich eine Identifikation des auf der Kante balancierenden Würfels durchzuführen. Hierbei werden aber die Parameter der diskreten Übertragungsfunktionen bestimmt, welche keine physikalische Größen repräsentieren. Somit ist es nicht möglich diese Ergebnisse auf den Fall des Würfels auf einer Ecke zu übertragen. Es bestehen zwar methoden für die Parameterschätzung kontinuierlicher Systeme, allerdings sind diese mit zusätzlichen Problemen verbunden [Unbehauen, S. 189 ff.]. Des weiteren ist es schwierig einen Versuch zu finden mit dem alle physikalischen Größen identifiziert werden können, die für das Balancieren auf einer Ecke relevant sind.

Der hier entwickelte Regler stellt eine Lösung für dieses Problem dar, da Methoden für die Parameterschätzung im geschlossenen Regelkreis bestehen [Unbehauen, S. 126 ff.]. Hierbei wird der Stellvektor aus der Summe des Reglergesetzes und eines Testsignales berechnet. Als Testsignal keine beispielsweise eine harmonische Schwingung verwendet werden deren Frequenz variiert wird um die volle Bandbreite des Systems abzudecken. Des weiteren kann die Amplitude des Testsignals variiert werden um den Einfluss der Nichtlinearitäten abzuschätzen [Unbehauen, S. 509 ff.].

Im Anschluss kann das hier vorgestellt Entwurfsverfahren für den Regler wiederholt werden. An dieser Stelle ist zu erwarten, dass sich eine deutliche Verbesserung des geschlossenen Regelkreises zeigt. 

Des weiteren wurde in dieser Arbeit lediglich ein lineares deterministisches Modell genutzt. Die Grenzen dieses Vorgehens wurden sowohl bei dem Entwurf des Reglers als auch des Filterkonzeptes ersichtlich. Die Stellgrößenbeschränkung wurde nur indirekt bei dem Reglerentwurf beachtet. Eine effiziente Lösung stellt entweder ein Sättigungsregler [Adamy ] oder ein allgemeines Anti-Windup [Ortseifen] dar.

Für die Verbesserung des Filters kann bei der Systemidentifikation eine Modellstruktur gewählt werden, die ein System- und Messrauschen beinhaltet [Unbehauen, S.59 ff.]. An Hand dieses Modells kann ein Kalman-Filter entwickelt werden um die Signalgüte zu erhöhen. Ein Vorteil dieser Methode ist, dass das Kalman-Filter auf das nichtlineare Systeme erweitert werden kann [Adamy, S.503 ff.] und smoit nicht auf einen Arbeitspunkt beschränkt ist.

Die Bearbeitung der weiteren Aufgaben wird durch diese Arbeit in der Hinsicht erleichtert, dass einerseits die Infrastruktur wiederverwendet werden kann. Beispielsweise muss der Signalfluss des geschlossenen Regelkreises für die Systemidentifkation lediglich um ein Testsignal erweitert werden. Der hierfür nötige Implementieraufwand ist vernachlässigbar, weshalb der Fokus der folgenden Arbeit allein auf dem systemtheoretischen Teilgebiet liegt. Des weiteren stellt diese Arbeit ein funktionierendes Gesamtkonzept vor. Somit können in weiteren Arbeiten einzelne Komponenten, wie zum Beispiel die Regler- oder Filtermethodik, bearbeitet und an dem bestehenden Gesamtsystem miteinander verglichen werden. Dadurch können die einzelnen Aufgaben parallel und fokussiert auf ein Teilgebiet bearbeitet werden.