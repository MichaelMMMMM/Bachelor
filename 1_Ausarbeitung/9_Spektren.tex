\section{Herleitung/Interpretation Fourier-Spektren}
Die Beurteilung der dynamischen Störungen erfolgt über eine Spektralanalyse. Hierfür wird die DFT der Messgrößen berechnet. Um die Interpretation der DFT-Spektren zu ermöglichen, wird zuerst der Zusammenhang der DFT und der Fourier-Transformation des zu Grunde liegendem Signal hergeleitet.
Die Fourier-Transformation basiert auf der Theorie, dass alle zeitkontinuierlichen Signale aus einer Summe von harmonischen Schwingungen synthetisiert werden können. Der Betrag der Fourier-Transformation gibt die Amplituden der beteiligten Schwingungen wieder. Nun ergibt sich die Frage wie das Spektrum eines zeitdiskreten Signals, das über einen begrenzten Zeitraum abgetastet wird, berechnet werden kann. Hierfür wird zunächst die komplexe harmonische Schwingung $x_n$ mit der Frequenz $f$ betrachtet, welche mit einer Frequenz $f_a$ ($T_a=1/f_a$) abgetastet wird. Mit Hilfe der normierten Kreisfrequenz $\Omega=2\cdot \pi \cdot f/f_a$, welche das Verhältnis der Schwingungsfrequenz und der Abtastrate darstellt, ergibt sich die folgende Funktion für $x_n$.

\begin{equation}
x_n = e^{j\cdot 2 \cdot \pi \cdot  n \cdot \frac{f}{f_a}} = e^{j\cdot n\cdot \Omega}
\end{equation} 

Wenn nun ein System, welches die diskrete Impulsantwort $h_n$ besitzt, von dem Signal $x_n$ angeregt wird, ergibt sich der folgende Zusammenhang für dessen Ausgangssignal $y_n$.

\begin{equation}
y_n = h_n * x_n = \sum^{\infty}_{k=-\infty}h_k \cdot e^{j\cdot\Omega\cdot(n-k)} = e^{j\cdot n\cdot \Omega} \cdot \sum^{\infty}_{k=-\infty} h_k \cdot e^{-j\cdot \Omega \cdot k}
\end{equation}

Hieraus folgt, dass die Antwort eines Systems auf eine diskrete harmonische Schwingung als Produkt der Schwingung mit dem Term $\sum h_k \cdot e^{-j\cdot \Omega \cdot k}$. Dieser kann somit als Systemverstärkung einer komplexen Eingangsschwingung mit der normierten Frequenz $\Omega$ interpretiert werden. Diese Systemeigenschaft entspricht der Aussage der gewöhnlichen Fourier-Transformierten einer zeitkontinuierlichen Übertagungsfunktion. Deshalb gilt für die DTFT (Discrete-Time-Fourier-Transform) eines Signals $x_n$:
\begin{equation}
X_{DTFT}(\Omega) = \sum^{\infty}_{k=-\infty} x_k \cdot e^{-j\cdot \Omega \cdot k}
\end{equation}
Hierbei ist zu beachten, dass die DTFT lediglich in Abhängikeit der normierten Kreisfrequenz $\Omega$ berechnet werden kann. Dies ist eine Folge der Abtastung, da das zeitdiskrete Signal $x_n$ keine Information üben den zeitlichen Abstand seiner Stützstellen enthält. Somit können auch die absoluten Frequenzen des Spektrums nur mit Hilfe der Abtastrate $f_a$ rekonstruiert werden. Für den Zusammenhang zwischen dem FT-Spektrum $X(j\omega)$ des ursprünglichen, zeitkontinuierlichen Signal  und dem DTFT-Spektrum $X_DTFT(\Omega)$ des zeitdiskreten Signales :
\begin{equation}
\frac{1}{f_a}X_{DTFT}(\frac{\omega}{f_a}) = X(j\omega)
\end{equation}

Die Problematik der DTFT liegt darin, das es sich zwar um ein zeitdiskretes Signal handelt, dieses aber nach wie vor über einen unendlichen Wertebereich definiert ist. D.h. es liegt eine geschlossene Funktion vor. Somit ermöglicht die DTFT nicht die Berechnung der Spektralanteile eines Signales, wessen Abtastwerte nur über einen begrenzten Zeitraum bekannt ist. Zusätzlich ist die DTFT selbst eine kontinuierliche Funktion und somit nur schwierig auf digitalen Rechnern umsetzbar. Deshalb soll nun erläutert werden wie ein diskretes Spektrum eines abgetasteten Signals, welches nur über einen begrenzten Zeitraum definiert ist, berechnet werden kann. Hierfür wird das Signal $\hat{x}_n$ betrachtet, welches für die ersten $N$ Abtastwerte der komplexen harmonischen Schwingung $x_n$ entspricht und anschließend verschwindet.
\begin{equation}
\hat{x}_n = \begin{cases}
				x_n & n \in [0;N-1] \\
				0   & n \not\in [0;N-1]
				\end{cases}
\end{equation}
Somit ergibt sich für die DTFT von $\hat{x}_n$:
\begin{equation}
\hat{X}_{DTFT}(\Omega) = \sum^{\infty}_{k=-\infty}\hat{x}_k \cdot e^{-j\cdot k \cdot \Omega} = \sum^{N-1}_k=0 x_k \cdot e^{-j \cdot k \cdot \Omega}
\end{equation}
Wird nun die DTFT wiederum über $N$ Werte diskretisiert erhält man als Ergebnisse die diskrete Fourier-Transformation (DFT).
\begin{equation}
X_{DFT}(m) = \sum^{N-1}_{k=0} x_k \cdot e^{-j\cdot k \cdot \Omega_m} \hspace{35pt} \Omega_m = \frac{2\cdot \pi \cdot m}{N}
\end{equation}
Bei der Herleitung der DFT wird ersichtlich, dass das berechnete Spektrum nicht dem des zeitdiskreten Signales $x_n$ entspricht, sonderm dem Spektrum von $x_n$ multipliziert mit einem Rechteckimpuls, wessen Breite dem Beobachtungszeitraum entspricht. Die Auswirkungen dieser Fensterung auf das DFT-Spektrum werden als Leakage-Effekte bezeichnet. Da für gewöhnlich Informationen über das Spektrum des ursprünglichen, zeitkontinuerlichen Signals gesucht sind müssen einerseits die Leakage-Effekte minimiert werden und der Zusammenhang zwischen FT-Spektrum und DFT-Spektrum hergestellt werden. Falls ein periodisches Signal $x(t)$ wird über einen Zeitraum $T$ beobachtet wird und $T$ kein Vielfaches der Periodenlänge des Signales ist so entstehen Signalsprünge am Ende der Beobachtung. Diese Sprüngen führen zu spektrale Überlappungen, welche wiederum das DFT-Spektrum verfälschen. Deshalb ist der Beobachtungszeitraum mit der Schwingung eines periodischen Signales zu synchronisieren. Des weiteren können Leakage-Effekte minimiert werden indem ein möglichst großer Beobachtungszeitraum gewählt wird. Ist dies der Fall gilt folgender Zusammenahng für Spektralanteile, welche keinen unendlichen Wert besitzen.
\begin{equation}
X_{DFT}(m) \approx f_a \cdot X(j\cdot m \cdot \Delta \omega) \hspace{35pt} \Delta\omega = 2\cdot \pi \cdot \frac{f_a}{N}
\end{equation}
Falls das Spektrum $X(j\cdot\omega)$ an der Stelle $m\cdot \Delta \omega$ einen Dirac-Impuls-Anteil besitzt so gilt für den komplexen Fourierkoeffizienten $c_k$ des Signales an dieser Frequenz:
\begin{equation}
X_{DFT}(m) = N \cdot c_k
\end{equation}